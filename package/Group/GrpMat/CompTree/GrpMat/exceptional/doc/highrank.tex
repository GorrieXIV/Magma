%******************************************************************************
%
%    highrank.tex Gary Seit'z exceptional groups algorithm
%
%    File      : $HeadURL:: https://subversion.sfac.auckland.ac.nz/svn/prj_m#$:
%    Author    : Henrik B��rnhielm
%    Dev start : 2008-11-01
%
%    Version   : $Revision:: 1345                                            $:
%    Date      : $Date:: 2008-11-02 18:36:09 +1100 (Sun, 02 Nov 2008)        $:
%    Last edit : $Author:: jbaa004                                           $:
%
%    $Id:: highrank.tex 1345 2008-11-02 07:36:09Z jbaa004                    $:
%
%******************************************************************************

\documentclass[twoside,a4paper,reqno,psamsfonts]{amsart}
%\usepackage[swedish,english]{babel} 
\usepackage[latin1]{inputenc} % svenska tecken skall tolkas
%\usepackage{amsmath}
%\usepackage{amsfonts} % for \mathbb
%\usepackage{amsthm}
\usepackage{amscd}
\usepackage{amsopn}
\usepackage{amstext}
\usepackage{amsxtra}
\usepackage{amssymb}
\usepackage{stmaryrd}
\usepackage{upref}
\usepackage{textcase} % For correct case of swedish chars
%\usepackage[algo2e,ruled,linesnumbered,algosection]{algorithm2e} % for typesetting algorithms
%\usepackage{clrscode}
%\renewcommand{\Comment}{$\hspace*{-0.075em} //$ } % Use // for comments
\usepackage{bm} % for \boldsymbol
%\usepackage{draftcopy}
\usepackage{graphicx}
\usepackage{setspace}
\frenchspacing

%\numberwithin{section}{chapter}
\numberwithin{equation}{section}
\numberwithin{figure}{section}
%\numberwithin{algorithm}{chapter}

%\newcounter{algorithm}
%\numberwithin{algorithm}{section}
%\renewcommand{\thealgorithm}{\arabic{section}.\arabic{algorithm}}

\theoremstyle{plain}
\newtheorem{thm}{Theorem}[section]
\newtheorem{lem}[thm]{Lemma}
\newtheorem{cl}[thm]{Corollary}
\newtheorem{pr}[thm]{Proposition}
\newtheorem{axiom}[thm]{Axiom}
\newtheorem{conj}[thm]{Conjecture}

\theoremstyle{definition}
\newtheorem{deff}[thm]{Definition}

\theoremstyle{remark}
\newtheorem{note}[thm]{Note}
\newtheorem{rem}[thm]{Remark}

%\renewcommand{\thetheorem}{Theorem~\arabic{chapter}.\arabic{theorem}}
%\renewcommand{\thepr}{Proposition~\arabic{chapter}.\arabic{pr}}
%\renewcommand{\thedeff}{Definition~\arabic{chapter}.\arabic{deff}}
\providecommand{\abs}[1]{\left\lvert #1 \right\rvert}
\providecommand{\norm}[1]{\left\lVert #1 \right\rVert}
\providecommand{\ceil}[1]{\left\lceil #1 \right\rceil}
\providecommand{\floor}[1]{\lfloor #1 \rfloor}
\providecommand{\set}[1]{\left\lbrace #1 \right\rbrace}
\providecommand{\gen}[1]{\left\langle #1 \right\rangle}
%\providecommand{\ord}[1]{\operatorname{ord}( #1 )}
\providecommand{\Sym}[1]{\operatorname{Sym}( #1 )}
\renewcommand{\Pr}[1]{\operatorname{Pr}[ #1 ]}
%\renewcommand{\char}[1]{\operatorname{char}[ #1 ]}

\newcommand{\field}[1]{\mathbb{#1}}
\newcommand{\vect}[1]{\boldsymbol{\mathrm{#1}}}
\newcommand{\N}{\field{N}}
\newcommand{\Z}{\field{Z}}
\newcommand{\R}{\field{R}}
\newcommand{\Q}{\field{Q}}
\newcommand{\OO}{\field{O}}
\newcommand{\K}{\field{K}}
\newcommand{\A}{\field{A}}
\newcommand{\F}{\field{F}}
\newcommand{\PS}{\field{P}}
\newcommand{\GAP}{\textsf{GAP}}
\newcommand{\MAGMA}{\textsc{Magma}}

% for Cayley graphs
\newcommand{\C}{\mathcal{C}}
\newcommand{\OV}{\mathcal{O}}

\DeclareMathOperator{\sgd}{sgd}
\DeclareMathOperator{\mgm}{mgm}
\DeclareMathOperator{\sgn}{sgn}
\DeclareMathOperator{\GL}{GL}
\DeclareMathOperator{\PGL}{PGL}
\DeclareMathOperator{\GF}{GF}
\DeclareMathOperator{\IM}{Im}
\DeclareMathOperator{\RE}{Re}
\DeclareMathOperator{\I}{Id}
%\DeclareMathOperator{\OR}{O}
\DeclareMathOperator{\SL}{SL}
\DeclareMathOperator{\GO}{GO}
\DeclareMathOperator{\SO}{SO}
\DeclareMathOperator{\Sz}{Sz}
\DeclareMathOperator{\Sp}{Sp}
\DeclareMathOperator{\SU}{SU}
\DeclareMathOperator{\chr}{char}
\DeclareMathOperator{\Aut}{Aut}
\DeclareMathOperator{\Alt}{Alt}
\DeclareMathOperator{\PSL}{PSL}
\DeclareMathOperator{\PPSL}{(P)SL}
\DeclareMathOperator{\diag}{diag}
\DeclareMathOperator{\Tr}{Tr}
\DeclareMathOperator{\SLP}{\mathtt{SLP}}
\DeclareMathOperator{\G2}{{^2}G_2}
\DeclareMathOperator{\LargeRee}{{^2}F_4}
\DeclareMathOperator{\Ree}{Ree}
\DeclareMathOperator{\Gal}{Gal}
\DeclareMathOperator{\Norm}{N}
\DeclareMathOperator{\Cent}{C}
\DeclareMathOperator{\Zent}{Z}
\DeclareMathOperator{\EndR}{End}
\DeclareMathOperator{\Hom}{Hom}
\DeclareMathOperator{\Ker}{Ker}
\DeclareMathOperator{\cln}{{:}}
\DeclareMathOperator{\O2}{O_2}
\DeclareMathOperator{\Op}{O_p}
\DeclareMathOperator{\RP}{\bf{RP}}
\DeclareMathOperator{\NP}{\bf{NP}}
\DeclareMathOperator{\PP}{\bf{P}}
\DeclareMathOperator{\coRP}{\bf{co-RP}}
\DeclareMathOperator{\ZPP}{\bf{ZPP}}
\DeclareMathOperator{\Imm}{Im}
\DeclareMathOperator{\Mat}{Mat}
\DeclareMathOperator{\Dih}{D}
\DeclareMathOperator{\4F}{F_4}
\DeclareMathOperator{\6E}{E_6}
\DeclareMathOperator{\2E6}{{^2}E_6}
\DeclareMathOperator{\7E}{E_7}
\DeclareMathOperator{\8E}{E_8}


\newcommand{\OR}[1]{\operatorname{O} \bigl( #1 \bigr)}
%\newcommand{\cln}{\operatorname{O} ( #1 )}

\title{Constructive recognition of exceptional groups}

\author{Henrik B\"a\"arnhielm}
\address{Department of Mathematics \\ University of Auckland \\ Auckland \\ New Zealand}
\urladdr{http://www.math.auckland.ac.nz/\textasciitilde henrik/}
\email{henrik@math.auckland.ac.nz}

\begin{document}

\begin{abstract}
Gary Seitz' algorithm for finding standard generators in exceptional groups of Lie rank at least 2.
\end{abstract}

\maketitle

\section{The algorithm}

The input is $G = \gen{X} \leqslant \GL(d^{\prime}, q^{\prime})$ where $G \cong S$ and
$S$ is $\4F(q), \6E(q), \2E6(q), \7E(q)$ or $\8E(q)$. This means that
$S \leqslant \GL(d, q)$ is the standard copy of one
of these groups, a chosen copy in the natural representation. We assume that
$G$ has been named using \cite{MR2352733} and
\cite{general_recognition}, so it is known which group $S$ is, and the
defining field $\F_q$ is known, with a fixed primitive
element $\omega$. We require that $q$ is odd.

\begin{table}[hb!]
\begin{tabular}{c|c|c|c|c}
$S$ & $d$ & $r$ & $\hat{H}_2$ & $\hat{C}_2$ \\
\hline
$\4F(q)$ & $26$ & $(q - 1)(q^3 + 1)$ & $\Sp(6, q)$ & $2.\Omega(9, q)$ \\
$\6E(q)$ & $27$ & $(q + 1) / (3, q - 1)$ or $(q^3 + 1)(q^2 + q + 1) / (3, q - 1)$ & $\SL(6, q)$ & See table \ref{tbl:e6_data} \\
$\2E6(q)$ & $27$ & $(q + 1) / (3, q + 1)$ or $(q^3 + 1)(q^2 + q + 1) / (3, q + 1)$ & $\SU(6, q)$ & See table \ref{tbl:e6_data} 
\end{tabular}
\caption{Group data}
\label{tbl:group_data}
\end{table}

\begin{table}[hb!]
\begin{tabular}{c|c|c}
$S$ & $q \pmod{12}$ & $\hat{C}_2$ \\
\hline
$\6E(q)$ & 1 & $(\Cent_{(q - 1) / 3} \circ 4 . \Omega^{+}(10, q)).4$ \\
$\6E(q)$ & 5 & $(\Cent_{(q - 1)} \circ 4 . \Omega^{+}(10, q)).4$ \\
$\6E(q)$ & 7 & $(\Cent_{(q - 1) / 3} \circ 2 . \Omega^{+}(10, q)).2$ \\
$\6E(q)$ & 11 & $(\Cent_{(q - 1)} \circ 2 . \Omega^{+}(10, q)).2$ \\
$\2E6(q)$ & 1 & $(\Cent_{(q + 1)} \circ 2 . \Omega^{-}(10, q)).2$ \\
$\2E6(q)$ & 5 & $(\Cent_{(q + 1) / 3} \circ 2 . \Omega^{-}(10, q)).2$ \\
$\2E6(q)$ & 7 & $(\Cent_{(q + 1)} \circ 4 . \Omega^{-}(10, q)).4$ \\
$\2E6(q)$ & 11 & $(\Cent_{(q + 1) / 3} \circ 4 . \Omega^{-}(10, q)).4$ \\
\end{tabular}
\caption{Data for $\6E(q)$ and $\2E6(q)$}
\label{tbl:e6_data}
\end{table}

Some objects are given in Table \ref{tbl:group_data}. $\Omega^{\epsilon}(d, q)$ refers to the simple orthogonal groups (true?), so $2.\Omega(9, q)$ is the subgroup of index $2$ in $\SO(9, q)$.

\begin{enumerate}
\item By random search, find an element of order $r$. Take a suitable
  power of it to obtain an involution $j_1$. Construct the centraliser
  $C_1 = \Cent_G(j_1)$ using \cite{bray00}.
\item From \cite{parkerwilson06} we know that with high probability,
  $C_1$ is a central or direct product of $H_1 \cong \hat{H}_1 = \SL(2, q)$ and
  $H_2 \cong \hat{H}_2$, possibly with decorations. Construct a
  composition tree for $C_1$ in order to verify that its structure
  (possibly use \cite{general_recognition} instead).
\item Using the algorithms in \cite[Section $5$]{babaibeals01} on
  $C_1^{\prime}$, obtain sets $Y_1, Y_2 \subset C_1$ such that $H_1 =
  \gen{Y_1}$ and $H_2 = \gen{Y_2}$. Name the groups using \cite{general_recognition} in order to verify their structures.
\item Construct composition trees for $H_1$ and $H_2$. The
  composition trees provide effective isomorphisms $\phi_1 : H_1 \to
  \hat{H}_1$ and $\phi_2 : H_2 \to \hat{H}_2$ with effective inverses.
\item Define the following elements of $\hat{H}_1$:
\begin{align}
\hat{s}_1 &= \begin{bmatrix}
0 & 1 \\
-1 & 0 \\
\end{bmatrix}, &
\hat{t}_1 &= \begin{bmatrix}
1 & 1 \\
0 & 1 \\
\end{bmatrix}, &
\hat{\delta}_1 &= \begin{bmatrix}
\omega^{-1} & 0 \\
0 & \omega \\
\end{bmatrix}, 
\end{align}
and let $s_1, t_1, \delta_1$ be their images in $G$ under
$\phi_1^{-1}$. 
\item Among the standard generators of $\hat{H}_2$, defined in
  \cite{classical_recognise}, there are elements $\hat{s}_2$,
  $\hat{t}_2$ and $\hat{\delta}_2$ such that the bijection $\hat{s}_1
  \mapsto \hat{s}_2, \hat{t}_1 \mapsto \hat{t}_2, \hat{\delta}_1
  \mapsto \hat{\delta}_2$ extends to an isomorphism of $\SL(2, q)$.
  Let $s_2, t_2, \delta_2$ be their images in $G$ under $\phi_2^{-1}$ and let $H_3 = \gen{s_2, t_2, \delta_2}$.
\item Let $j_2 = s_2^2$ (possibly different if $S$ is $\7E(q)$ or
  $\8E(q)$). By construction, $j_1 \in H_1$ and $j_2 \in H_2$ and
  $[H_1, H_2] = 1$. Hence $\gen{j_1, j_2}$ is a Klein $4$-group.
  Moreover, $C_2 = \Cent_G(j_1 j_2) \cong \hat{C}_2$ (proof needed).
  Construct $C_2$ using \cite{bray00} and construct a composition tree
  for $C_2$.
\item Now $\Cent_{C_2}(j_1) \cong H_1 . H_3 . C_4$ where $C_4$ is a
  classical group, and there exists an involution $\mu \in C_2$ such
  that $H_1^{\mu} = H_3$ (proof needed). Find this involution using
  the algorithms described below.
\item The standard generators of $G$ are now the standard generators
  of $H_2$, defined in \cite{classical_recognise} and obtained as
  images under $\phi_2^{-1}$ of the standard generators of
  $\hat{H}_2$, and $\mu$. Clearly, $\mu \notin H_2$, and these standard generators generate the whole of $G$ (proof needed).
\end{enumerate}

Note that we obtain the generators of $C_1$ and $C_2$ as $\SLP$s in
the generators of $G$. When we construct $C_1^{\prime}$ we obtain its
generators as $\SLP$s in the generators of $C_1$. The algorithms in
\cite[Section $5$]{babaibeals01} then provide $\SLP$s for the
generators of $H_2$ in the generators of $C_1^{\prime}$. The
composition trees for $H_2$ and $C_2$ gives the ability to express any
of their elements as $\SLP$s in their generators. Hence we can obtain
the standard generators of $H_2$ as well as $\mu$ as $\SLP$s in the
generators of $G$.

\section{Finding the involution $\mu$}

\subsection{The case $S = \4F(q)$}
We have a composition tree for $C_2$. This provides an isomorphism(?)
$\psi : C_2 \to \hat{C}_2 \leqslant \GL(9, q)$. Both $H_1$ and $H_3$
lie in $C_2$. Let $\bar{H_1} = \gen{\psi(s_1), \psi(t_1),
  \psi(\delta_1)}$, $\bar{H_3} = \gen{\psi(s_2), \psi(t_2),
  \psi(\delta_2)}$ and let $H = \gen{\bar{H}_1, \bar{H}_3}$. The
module $V = \F_q^9$ of $\hat{C}_2$ has an associated symmetric
bilinear form which is preserved by $\hat{C}_2$. Also, $V\vert_H$ is a
direct sum of a submodule $U$ of dimension $4$ and six $1$-dimensional
submodules (proof needed). Now define another $4$-dimensional
$H$-module $W$, where $\hat{H}_1$ and $\hat{H}_3$ act with the
following matrices.
\begin{align}
\bar{s}_1 &= \begin{bmatrix}
0 & 1 & 0 & 0 \\
-1 & 0 & 0 & 0 \\
0 & 0 & 0 & -1 \\
0 & 0 & -1 & 0 \\
\end{bmatrix}, &
\bar{t}_1 &= \begin{bmatrix}
1 & 1 & 0 & 0 \\
0 & 1 & 0 & 0 \\
0 & 0 & 1 & 1 \\
0 & 0 & 0 & 1 \\
\end{bmatrix}, &
\bar{\delta}_1 &= \begin{bmatrix}
\omega^{-1} & 0 & 0 & 0 \\
0 & \omega & 0 & 0 \\
0 & 0 & \omega^{-1} & 0 \\
0 & 0 & 0 & \omega \\
\end{bmatrix}; \\
\bar{s}_2 &= \begin{bmatrix}
0 & 0 & 1 & 0 \\
0 & 0 & 0 & 1 \\
-1 & 0 & 0 & 0 \\
0 & -1 & 0 & 0 \\
\end{bmatrix}, &
\bar{t}_2 &= \begin{bmatrix}
1 & 0 & 1 & 0 \\
0 & 1 & 0 & 1 \\
0 & 0 & 1 & 0 \\
0 & 0 & 0 & 1 \\
\end{bmatrix}, &
\bar{\delta}_2 &= \begin{bmatrix}
\omega^{-1} & 0 & 0 & 0 \\
0 & \omega^{-1} & 0 & 0 \\
0 & 0 & \omega & 0 \\
0 & 0 & 0 & \omega \\
\end{bmatrix} 
\end{align}
Clearly, $U$ and $W$ are isomorphic $H$-modules. Use the algorithm in
\cite{meataxe} to obtain a change of basis matrix $c$ that conjugates
$W$ to $U$. Let $p \in \GL(4, q)$ be the permutation matrix
corresponding to $(2, 3) \in \Sym{4}$. Now define an involution
$\hat{\mu}$ which acts as $p^c$ on $U$ and as $-\I_6$ on $U^{\perp}$.
By construction this preserves the form on $V$, conjugates $\bar{H}_1$
to $\bar{H}_3$ and has determinant $1$. Return \texttt{false} if the
spinor norm of $\hat{\mu}$ is not $0$.

Let $\mu = \psi^{-1}(\hat{\mu})$. Now $\abs{\mu} \in {2, 4}$ (proof
needed). In the latter case, let $\mu = \psi^{-1}(\hat{\mu}) s_1^2$.
Then $\mu \in C_2$ is an involution (proof needed) that conjugates
$H_1$ to $H_3$.

\bibliographystyle{amsalpha}
\bibliography{mgrp}

\end{document}
