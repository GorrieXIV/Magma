% $Id:: f4.tex 1241 2008-10-25 14:05:02Z jbaa004                             $:

\def\SL{{\rm SL}}
\def\Sp{{\rm Sp}}
\def\GF{{\rm GF}}

\magnification 1200
\centerline{\bf Gary Seitz' $F_4$ algorithm}
\medskip
Let $G$ be isomorphic to $F_4(q)$.  To find standard generators.

1.  Find, by random search, an element of $G$ that powers to an involution $j$ whose
centraliser is isomorphic to $(\SL_2\circ\Sp_6).2$, and construct the centraliser.

2.  Extract $\SL_2$ and $\Sp_6$ from the above centraliser, and find standard generators
for these groups.  These will be:

For $\SL_2$ the generators $\{s,t,\delta\}$ where
$$s=\left(\matrix{0&1\cr-1&0\cr}\right)\quad t=\left(\matrix{1&1\cr0&1\cr}\right)
\quad \delta=\left(\matrix{\omega^{-1}&0\cr0&\omega\cr}\right),$$
where $\omega$ is a primitive element of $\GF(q)$.  This primitive element is fixed; a
change of $\omega$ produces an inequivalent set of generators.  $\delta$ is omitted if
$q$ is prime.

For $\Sp_6$ there are more standard generators; but the first three (two if $q$ is prime)
are the same matrices (w.r.t. a different basis, of course).

To distinguish the standard generators of $\SL_2$ from the first three of $\SL_6$
write them as $\{s_S,t_S,\delta_S\}$ and $\{s_T,t_T,\delta_T\}$ respectively.

3.  Set $k={s_S}^2j$, so $kj=s_S^2$ (or the other way round; it should always be the same way)
so that $C:=C_G(kj)$ is isomorphic to $C_2.\Omega_9$.

4.  Note that $S=\langle s_S,t_S,\delta_S\rangle$ and $T=\langle s_T,t_T,\delta_T\rangle$ are
subgroups of $C$.  The image $\mu$ of the final generator in $\Omega_9$ will be defined as follows. 
$\langle S,T\rangle$ maps to a subgroup of $\Omega_9$ that centralises a subspace $U$ of
dimension 5, and acts as $\SL_2\circ\SL_2$ on the orthogonal complement $W$ of this space.
To specify $\mu$ it suffices to specify its action on $U$ and $W$.  The action on $W$ is defined as follows.  Find a basis for $W$ with respect to which the standard bases act as follows.

$$s_S=\left(\matrix{0&1&0&0\cr-1&0&0&0\cr0&0&0&1\cr0&0&-1&0\cr}\right)\quad
t_S=\left(\matrix{1&1&0&0\cr0&1&0&0\cr0&0&1&1\cr0&0&0&1\cr}\right)\quad
\delta_S=\left(\matrix{\omega^{-1}&0&0&0\cr0&\omega&0&0\cr0&0&\omega^{-1}&1\cr0&0&0&\omega\cr}\right)$$
$$s_T=\left(\matrix{0&0&1&0\cr0&0&0&1\cr-1&0&0&0\cr0&-1&0&0\cr}\right)\quad
t_T=\left(\matrix{1&0&1&0\cr0&1&0&1\cr0&0&1&0\cr0&0&0&1\cr}\right)\quad
\delta_T=\left(\matrix{\omega^{-1}&0&0&0\cr0&\omega^{-1}&0&0\cr0&0&\omega&0\cr0&0&0&\omega\cr}\right).$$

This is done by applying module isomorphism to $W$ as a module for $\langle S, T\rangle$.

The action of $\mu$ on $W$ is now defined as follows.  Fix the first and fourth basis vectors, and
interchange the second and third.

The action on $U$ is multiplication by $-1$ (to get the determinant right).

Now lift $\mu$ from $\Omega_9$ to $2.\Omega_9$.  I don't whether it matters which of the two lifts
you use.  You can get from one to the other by multiplying by the generator of the $C_2$, as
found by composition tree.

\bye