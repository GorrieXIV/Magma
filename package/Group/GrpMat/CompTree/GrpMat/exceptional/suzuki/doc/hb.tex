%******************************************************************************
%
%    hb.tex      Magma Suzuki package handbook
%
%    File      : $HeadURL:: https://subversion.sfac.auckland.ac.nz/svn/prj_m#$:
%    Author    : Henrik B��rnhielm 
%    Dev start : 2008-06-08
%
%    Version   : $Revision:: 10                                              $:
%    Date      : $Date:: 2008-04-11 16:31:41 +1200 (fre, 11 apr 2008)        $:
%    Last edit : $Author:: eobr007                                           $:
%
%    $Id:: design.tex 10 2008-04-11 04:31:41Z eobr007                        $:
%
%******************************************************************************

\documentclass[twoside,a4paper,reqno]{amsart}
%\usepackage[swedish,english]{babel} 
%\usepackage[latin1]{inputenc} % svenska tecken skall tolkas
\usepackage{graphicx}
%\usepackage{pictex}
\usepackage{amsmath}
\usepackage{amsfonts} % for \mathbb
\usepackage{amsthm}
\usepackage{amscd}
\usepackage{amsopn}
\usepackage{amstext}
\usepackage{amsxtra}
%\usepackage{stmaryrd}
\usepackage{amssymb}
%\usepackage{amsbsy}
\usepackage{textcase} % For correct case of swedish chars

\usepackage{bm} % for \boldsymbol
%\usepackage{draftcopy}
\frenchspacing

\numberwithin{equation}{section}
\numberwithin{figure}{section}
%\numberwithin{algorithm}{chapter}

\theoremstyle{plain}
\newtheorem{thm}{Theorem}[section]
\newtheorem{lem}[thm]{Lemma}
\newtheorem{cl}[thm]{Corollary}
\newtheorem{prop}[thm]{Proposition}
%\newtheorem{axiom}[theorem]{Axiom}
\newtheorem{conj}[thm]{Conjecture}

\theoremstyle{definition}
\newtheorem{deff}[thm]{Definition}

\theoremstyle{remark}
%\newtheorem{note}[theorem]{Note}
\newtheorem{rem}[thm]{Remark}

\newcounter{algorithm2e}
%\renewcommand{\thealgorithm2e}{\algorithmcfname~\arabic{section}.\arabic{algorithm2e}}

%\renewcommand{\thetheorem}{Theorem~\arabic{chapter}.\arabic{theorem}}
%\renewcommand{\thepr}{Proposition~\arabic{chapter}.\arabic{pr}}
%\renewcommand{\thedeff}{Definition~\arabic{chapter}.\arabic{deff}}
\providecommand{\abs}[1]{\left\lvert #1 \right\rvert}
\providecommand{\norm}[1]{\left\lVert #1 \right\rVert}
\providecommand{\ceil}[1]{\left\lceil #1 \right\rceil}
\providecommand{\floor}[1]{\lfloor #1 \rfloor}
\providecommand{\set}[1]{\left\lbrace #1 \right\rbrace}
\providecommand{\gen}[1]{\left\langle #1 \right\rangle}
%\providecommand{\ord}[1]{\operatorname{ord}( #1 )}
\providecommand{\Sym}[1]{\operatorname{Sym}( #1 )}
\renewcommand{\Pr}[1]{\operatorname{Pr}[ #1 ]}
%\renewcommand{\char}[1]{\operatorname{char}[ #1 ]}

\newcommand{\field}[1]{\mathbb{#1}}
\newcommand{\vect}[1]{\boldsymbol{\mathrm{#1}}}
\newcommand{\N}{\field{N}}
\newcommand{\Z}{\field{Z}}
\newcommand{\R}{\field{R}}
\newcommand{\Q}{\field{Q}}
\newcommand{\K}{\field{K}}
\newcommand{\A}{\field{A}}
\newcommand{\F}{\field{F}}
\newcommand{\PS}{\field{P}}
\newcommand{\GAP}{\textsf{GAP}}
\newcommand{\MAGMA}{\textsc{Magma}}

% for Cayley graphs
\newcommand{\C}{\mathcal{C}}
\newcommand{\OV}{\mathcal{O}}

\DeclareMathOperator{\sgd}{sgd}
\DeclareMathOperator{\mgm}{mgm}
\DeclareMathOperator{\sgn}{sgn}
\DeclareMathOperator{\GL}{GL}
\DeclareMathOperator{\GF}{GF}
\DeclareMathOperator{\IM}{Im}
\DeclareMathOperator{\RE}{Re}
\DeclareMathOperator{\I}{Id}
%\DeclareMathOperator{\OR}{O}
\DeclareMathOperator{\SL}{SL}
\DeclareMathOperator{\GO}{GO}
\DeclareMathOperator{\SO}{SO}
\DeclareMathOperator{\Sz}{Sz}
\DeclareMathOperator{\Sp}{Sp}
\DeclareMathOperator{\Ree}{Ree}
\DeclareMathOperator{\chr}{char}
\DeclareMathOperator{\Aut}{Aut}
\DeclareMathOperator{\Alt}{Alt}
\DeclareMathOperator{\PSL}{PSL}
\DeclareMathOperator{\diag}{diag}
\DeclareMathOperator{\Tr}{Tr}
\DeclareMathOperator{\SLP}{\mathtt{SLP}}
\DeclareMathOperator{\Norm}{N}
\DeclareMathOperator{\Cent}{C}
\DeclareMathOperator{\Zent}{Z}

\newcommand{\OR}[1]{\operatorname{O} ( #1 )}

\title{Handbook for \textsc{Magma} Suzuki package}
\author{Henrik B\"a\"arnhielm}
\address{School of Mathematical Sciences \\ Queen Mary, University of London \\ Mile End Road \\ London E1 4NS \\ United Kingdom}
\date{2007-04-21}
\urladdr{http://www.maths.qmul.ac.uk/\textasciitilde hb/}
\email{h.baarnhielm@qmul.ac.uk}

\begin{document}


\maketitle
%\end{frontmatter}

The \textsc{Magma} Suzuki package primarily provides functionality for
constructive recognition and constructive membership testing of the Suzuki
groups $\mathrm{Sz}(q)$, with $q = 2^{2m + 1}$ for some $m > 0$. This
family of finite simple groups is already available in \textsc{Magma}
via the \texttt{Sz} intrinsic. It also provides routines for finding and conjugating Sylow and maximal subgroups of $\Sz(q)$.

The main intrinsics of the package are \texttt{RecogniseSz(GrpMat G)}
which performs constructive recognition of $G \cong \mathrm{Sz}(q)$,
\texttt{SzElementToWord(GrpMat G, GrpMatElt g)} which returns a
\texttt{GrpSLPElt} for $g$ in the generators of $G$, and
\texttt{SuzukiRecognition(GrpMat G)} which is a non-constructive test
for isomorphism between $G$ and $\mathrm{Sz}(q)$.
  
Sylow $p$-subgroups can be found using the \texttt{SuzukiSylow(GrpMat
  G, RngIntElt p)} intrinsic. To find conjugating elements between
Sylow $p$-subgroups, an intrinsic \texttt{SuzukiSylowConjugacy} is
provided. A list of representatives of the conjugacy classes of
maximal subgroups can be found using the
\texttt{SuzukiMaximalSubgroups(GrpMat G)} intrinsic, and the intrinsic
\texttt{SuzukiMaximalSubgroupsConjugacy} can be used to find
conjugating elements between maximal subgroups.

Presentations for $\mathrm{Sz}(q)$ can be handled via the
intrinsics \texttt{SzPresentation} and
\texttt{SatisfiesSzPresentation}.

There are a few verbose flags used in the package.
\begin{itemize}
\item \texttt{SuzukiGeneral}, for the general routines.
\item \texttt{SuzukiStandard}, for the routines related to the standard copy.
\item \texttt{SuzukiConjugate}, for the routines related to conjugation.
\item \texttt{SuzukiTensor}, for the routines related to tensor decomposition.
\item \texttt{SuzukiMembership}, for the routines related to membership testing.
\item \texttt{SuzukiCrossChar}, for the routines related to cross-characteristic representations.
\item \texttt{SuzukiSylow}, for the routines related to Sylow subgroups.
\item \texttt{SuzukiMaximals}, for the routines related to maximal subgroups.
\item \texttt{SuzukiTrick}, for the routines related to the double coset trick.
\item \texttt{SuzukiNewTrick}, for the routines related to the stabiliser trick.
\item \texttt{SuzukiElements}, for the routines related to element conjugacy.
\end{itemize}

All the flags can be set to values up to $10$, with higher values resulting in more output.

\section{Intrinsics}

\paragraph{}

\begin{verbatim}
RecogniseSz(G : parameters) : GrpMat -> BoolElt, Map, Map, Map, Map
\end{verbatim}
Parameters:
\begin{itemize}
\item[] \texttt{Verify}, \textsc{BoolElt}, \emph{Default : true}
\item[] \texttt{FieldSize}, \textsc{RngIntElt}
\end{itemize}

$G$ is absolutely irreducible and defined over minimal field.
Constructively recognise $G$ as a Suzuki group. If $G$ is isomorphic to $\mathrm{Sz}(q)$ where $q$ is the size of the defining field of $G$, then return:
\begin{enumerate}
\item Isomorphism from $G$ to $\mathrm{Sz}(q)$.
\item Isomorphism from $\mathrm{Sz}(q)$ to $G$.
\item Map from $G$ to the word group of $G$.
\item Map from the word group of $G$ to $G$.
\end{enumerate}
    
The isomorphisms are composed of maps that are defined by rules, so
\texttt{Function} can be used on each component, hence avoiding
unnecessary built-in membership testing.

The word group is the \texttt{GrpSLP} which is the parent of the
elements returned by \texttt{SzElementToWord}. In general this is not
the same as \texttt{WordGroup(G)}, but is created from it using
\texttt{AddRedundantGenerators}.
    
If \texttt{Verify} is true, then it is checked that $G$ is isomorphic to
$\mathrm{Sz}(q)$, using \texttt{SuzukiRecognition}, otherwise this is not checked. In that case, \texttt{FieldSize} must be set to the correct value of $q$.

Constructive recognition of $2.\mathrm{Sz}(8)$ is also handled.

The algorithms for constructive recognition are those of \cite{baarnhielm05} and \cite{sz_tensor_decompose}.

\paragraph{}

\begin{verbatim}
SzElementToWord(G, g) : GrpMat, GrpMatElt -> BoolElt, GrpSLPElt
\end{verbatim}

If $G$ has been constructively recognised as a Suzuki group,
and if $g$ is an element of $G$, then return \texttt{true} a \texttt{GrpSLPElt} from the word group of $G$ which evaluates to $g$, else return \texttt{false}.

This facilitates membership testing in $G$.

\paragraph{}

\begin{verbatim}
SzPresentation(q) : RngIntElt -> GrpFP, HomGrp
\end{verbatim}

If $q = 2^{2m + 1}$ for some $m > 0$, return a short presentation of
$\mathrm{Sz}(q)$ on the \textsc{Magma} standard generators,
\emph{i.e.} the generators returned by the \texttt{Sz} intrinsic.

Note that it is an open problem whether or not this is a presentation
of $\mathrm{Sz}(q)$ for every such $q$.

\paragraph{}

\begin{verbatim}
SatisfiesSzPresentation(G) : GrpMat -> BoolElt
\end{verbatim}

$G$ is constructively recognised as $\mathrm{Sz}(q)$ for some $q$.
Verify that it satisfies a presentation for this group.

\paragraph{}

\begin{verbatim}
SuzukiRecognition(G) : GrpMat -> BoolElt, RngIntElt
\end{verbatim}
Determine (non-constructively) if $G$ is isomorphic to $\mathrm{Sz}(q)$. The corresponding $q$ is also returned.
    
If $G$ is over a field of odd characteristic or has degree greater
than $4$, the Monte Carlo algorithm of \texttt{IdentifyLieType} is
used. If $G$ has degree $4$ and is over a field of characteristic $2$,
then a fast Las Vegas algorithm is used, described in
\cite{baarnhielm05}.

\paragraph{}

\begin{verbatim}
SuzukiIrreducibleRepresentation(F, twists : parameters) : 
FldFin, SeqEnum[RngIntElt] -> GrpMat
\end{verbatim}
Parameters:
\begin{itemize}
\item[] \texttt{CheckInput}, \textsc{BoolElt}, \emph{Default : true}
\end{itemize}

$F$ must have size $q = 2^{2m + 1}$ for some $m > 0$, and \emph{twists} should be a sequence of $n$ distinct integers in the range $[0 \ldots 2m]$.

Return an absolutely irreducible representation of $\mathrm{Sz}(q)$ of
dimension $4^n$, a tensor product of twisted powers of the copy
returned by the \texttt{Sz} intrinsic, where the twists are given by
the input sequence.

If \texttt{CheckInput} is true, then it is verified that $F$ and \emph{twists} satisfy the above requirements. Otherwise this is not checked.
\paragraph{}

\begin{verbatim}
SuzukiSylow(G, p) : GrpMat, RngIntElt -> GrpMat, SeqEnum
\end{verbatim}

If $G$ has been constructively recognised as a Suzuki group, and if
$p$ is a prime number, return a random Sylow $p$-subgroup $S$ of $G$.

Also returns a list of \texttt{GrpSLPElt} from the word group of $G$,
of the generators of $S$. If $p$ does not divide $\left\lvert G
\right\rvert$, then the trivial subgroup is returned.

\paragraph{}

\begin{verbatim}
SuzukiSylowConjugacy(G, R, S, p) : GrpMat, GrpMat, GrpMat, RngIntElt -> 
GrpMatElt, GrpSLPElt
\end{verbatim}

If $G$ has been constructively recognised as a Suzuki group, if $p$ is
a prime number and if $R$ and $S$ are Sylow $p$-subgroups of $G$, then
return an element $g$ of $G$ that conjugates $R$ to $S$. A
\texttt{GrpSLPElt} from the word group of $G$, that evaluates to $g$,
is also returned.

\paragraph{}

\begin{verbatim}
SuzukiMaximalSubgroups(G) : GrpMat -> SeqEnum, SeqEnum
\end{verbatim}

If $G$ has been constructively recognised as a Suzuki group, find a
list of representatives of the maximal subgroups of $G$.

Also returns lists of \texttt{GrpSLPElt} of the generators of the subgroups, from the word group of $G$.

\paragraph{}

\begin{verbatim}
SuzukiMaximalSubgroupsConjugacy(G, R, S) : GrpMat, GrpMat, GrpMat -> GrpMatElt, GrpSLPElt
\end{verbatim}

If $G$ has been constructively recognised as a Suzuki group and if $R$
and $S$ are conjugate maximal subgroups of $G$, then return an element
$g$ of $G$ that conjugates $R$ to $S$. A \texttt{GrpSLPElt} from the
word group of $G$, that evaluates to $g$, is also returned.

\paragraph{}

\begin{verbatim}
SzConjugacyClasses(G) : GrpMat -> SeqEnum
\end{verbatim}

If $G$ has been constructively recognised as a Suzuki group, return a list of conjugacy classes, using the same format as the \texttt{ConjugacyClasses} intrinsic.

\paragraph{}

\begin{verbatim}
SzClassRepresentative(G, g) : GrpMat, GrpMatElt -> GrpMatElt, GrpMatElt
\end{verbatim}

If $G$ has been constructively recognised as a Suzuki group, and $g
\in G$, return the conjugacy class representative $h$ of $g$, such
that $h$ is in the list returned by \texttt{SzConjugacyClasses}. Also
returns $c \in G$ such that $g^c = h$.

\paragraph{}

\begin{verbatim}
SzIsConjugate(G, g, h) : GrpMat, GrpMatElt, GrpMatElt -> BoolElt, GrpMatElt
\end{verbatim}

If $G$ has been constructively recognised as a Suzuki group, and $g, h
\in G$, determine if $g$ is conjugate to $h$. If so, return \texttt{true} and an element $c$ such that $g^c = h$, otherwise return \texttt{false}.

\paragraph{}

\begin{verbatim}
SzClassMap(G) : GrpMat -> Map
\end{verbatim}

If $G$ has been constructively recognised as a Suzuki group, return
its class map, as in the \texttt{ClassMap} intrinsic.

\section{Examples}

Example showing the basic features.
\begin{verbatim}
> // Let's try a conjugate of the the standard copy
> G := Sz(32);
> G ^:= Random(Generic(G));
> // perform non-constructive recognition
> flag, q := SuzukiRecognition(G);
> print flag, q eq 32;
true true
> // perform constructive recognition
> flag, iso, inv, g2slp, slp2g := RecognizeSz(G);
> print flag;
true
> // the explicit isomorphisms are defined by rules
> print iso, inv;
Mapping from: GrpMat: G to MatrixGroup(4, GF(2^5)) of order 32537600 = 2^10 * 
5^2 * 31 * 41 given by a rule [no inverse]
Mapping from: MatrixGroup(4, GF(2^5)) of order 32537600 = 2^10 * 5^2 * 31 * 41 
to GrpMat: G given by a rule [no inverse]
> // so we can use Function to avoid Magma built-in membership testing
> // we might not obtain the shortest possible SLP
> w := Function(g2slp)(G.1);
> print #w;
300
> // and the algorithm is probabilistic, so different executions will most
> // likely give different results
> ww := Function(g2slp)(G.1);
> print w eq ww;
false
> // the resulting SLPs are from another word group
> W := WordGroup(G);
> print NumberOfGenerators(Parent(w)), NumberOfGenerators(W);
7 3
> // but can be coerced into W
> flag, ww := IsCoercible(W, w);
> print flag;
true
> // so there are two ways to get the element back
> print slp2g(w) eq Evaluate(ww, UserGenerators(G));
true
> // an alternative is this intrinsic, which is better if the elements are not 
> // known to lie in the group
> flag, ww := SzElementToWord(G, G.1);
> print flag, slp2g(w) eq slp2g(ww);
true true
> // let's try something just outside the group
> H := Sp(4, 32);
> flag, ww := SzElementToWord(G, H.1);
> print flag;
false
> // in this case we will not get an SLP
> ww := Function(g2slp)(H.1);
> print ww;
false
> // we do indeed have a Suzuki group
> print SatisfiesSzPresentation(G);
true
> // finally, let's try 2.Sz(8)
> A := ATLASGroup("2Sz8");
> reps := MatRepKeys(A);
> G := MatrixGroup(reps[3]);
> print Degree(G), CoefficientRing(G);
40 Finite field of size 7
> // we can handle constructive recognition, although it takes some time
> time flag, iso, inv, g2slp, slp2g := RecognizeSz(G);
Time: 415.660
> print flag;
true
> // and constructive membership testing
> R := RandomProcess(G);
> g := Random(R);
> w := Function(g2slp)(g);
> // in this case the elements is determined up to a scalar
> print IsScalar(slp2g(w) * g^(-1));
true
\end{verbatim}

Example showing a case in large dimension.
\begin{verbatim}
> // let's try a 64-dimensional Suzuki group
> F := GF(2, 9);
> twists := [0, 3, 6];
> G := SuzukiIrreducibleRepresentation(F, twists);
> print Degree(G), IsAbsolutelyIrreducible(G);
64 true
> // conjugate it to hide the tensor product
> G ^:= Random(Generic(G));
> // and write it over a smaller field to make things difficult
> flag, GG := IsOverSmallerField(G);
> print flag, CoefficientRing(GG);
true Finite field of size 2^3
> // non-constructive recognition is harder in this case
> // and will give us the defining field size
> time print SuzukiRecognition(GG);
true 512
Time: 1.030
> // constructive recognition will decompose the tensor product
> time flag, iso, inv, g2slp, slp2g := RecogniseSz(GG);
Time: 2.170
> print iso;
Mapping from: GrpMat: GG to MatrixGroup(4, GF(2^9)) of order 2^18 * 5 * 7 * 13 *
37 * 73 * 109 given by a rule [no inverse]
> // constructive membership is again easy
> R := RandomProcess(GG);
> g := Random(R);
> time w := Function(g2slp)(g);
Time: 0.010
> // but SLP evaluation is harder in large dimensions
> time print slp2g(w) eq g;
true
Time: 0.040
> // we still have a Suzuki group
> time print SatisfiesSzPresentation(GG);
true
Time: 1.750
\end{verbatim}

Example showing a case in cross characteristic.

\begin{verbatim}
> // Let's try a Suzuki group in cross characteristic
> G := Sz(8);
> _, P := SuzukiPermutationRepresentation(G);
> // for example over GF(9)
> M := PermutationModule(P, GF(3, 2));
> factors := CompositionFactors(M);
> H := ActionGroup(factors[2]);
> // we actually end up with a group over GF(3)
> print IsAbsolutelyIrreducible(H);
true
> flag, G := IsOverSmallerField(H);
> print Degree(G), CoefficientRing(G);
64 Finite field of size 3
> // constructive recognition is easy in this case
> time flag, iso, inv, g2slp, slp2g := RecogniseSz(G);
Time: 0.650
> print iso;
Mapping from: GrpMat: G to MatrixGroup(4, GF(2^3)) of order 29120 = 2^6 * 5 * 7 
* 13 given by a rule [no inverse]
> // as well as constructive membership
> R := RandomProcess(G);
> g := Random(R);
> time w := Function(g2slp)(g);
Time: 0.000
> time print slp2g(w) eq g;
true
Time: 0.020
> // we still have a Suzuki group
> time print SatisfiesSzPresentation(G);
true
Time: 0.100
\end{verbatim}

Example showing a larger field case.

\begin{verbatim}
> // let's try a larger field
> q := 2^121;
> G := Sz(q);
> // let's try a conjugate
> G ^:= Random(Generic(G));
> // also move to another generating set
> G := DerivedGroupMonteCarlo(G);
> print NumberOfGenerators(G);
19
> // non-constructive recognition is now a bit harder
> time SuzukiRecognition(G);
true 2658455991569831745807614120560689152
Time: 0.090
> // and is your machine up for this?
> time flag, iso, inv, g2slp, slp2g := RecogniseSz(G);
Time: 291.160
> // each call to constructive membership is then easy
> R := RandomProcess(G);
> g := Random(R);
> time w := Function(g2slp)(g);
Time: 0.060
> // evaluating SLPs always takes some time
> time print slp2g(w) eq g;
true
Time: 1.470
\end{verbatim}

Example about Sylow subgroups.

\begin{verbatim}
> // let's try a small field
> q := 2^9;
> G := Sz(q);
> // let's try a conjugate
> G ^:= Random(Generic(G));
> // also move to another generating set
> G := DerivedGroupMonteCarlo(G);
> print NumberOfGenerators(G);
19
> // first we must recognise the group
> time flag, iso, inv, g2slp, slp2g := RecogniseSz(G);
Time: 0.100
> // try creating some Sylow subgroups
> p := Random([x[1] : x in Factorization(q - 1)]);
> print p;
73
> time R := SuzukiSylow(G, p);
Time: 0.020
> time S := SuzukiSylow(G, p);
Time: 0.000
> // that was easy, as is conjugating them
> time g, slp := SuzukiSylowConjugacy(G, R, S, p);
Time: 0.000
> time print R^g eq S;
true
Time: 0.000
> // in this case we also automatically get an SLP of the conjugating element
> print slp2g(slp) eq g;
true
> // those Sylow subgroups are cyclic, and we know the order
> print #R, NumberOfGenerators(R);
73 1
> // creating Sylow 2-subgroups is harder, since there are lots of generators
> time R := SuzukiSylow(G, 2);
Time: 0.040
> time S := SuzukiSylow(G, 2);
Time: 0.050
> print NumberOfGenerators(R), #R;
9 262144
> // but conjugation is easy
> time g, slp := SuzukiSylowConjugacy(G, R, S, 2);
Time: 0.000
> // verifying the conjugating element can be hard
> time print R^g eq S;
true
Time: 201.470
\end{verbatim}

Example about maximal subgroups.

\begin{verbatim}
> // let's try a small field
> m := 4;
> F := GF(2, 2 * m + 1);
> q := #F;
> print q;
512
> G := Sz(F);
> // let's try a conjugate
> G ^:= Random(Generic(G));
> // also move to another generating set
> G := DerivedGroupMonteCarlo(G);
> print NumberOfGenerators(G);
19
> // first we must recognise the group
> time flag, iso, inv, g2slp, slp2g := RecogniseSz(G);
Time: 0.140
> // try finding the maximal subgroups
> time l, slps := SuzukiMaximalSubgroups(G);
Time: 0.070
> // should have a parabolic, a torus normaliser, two Frobenius groups
> // and one smaller Sz
> print #l;
5
> // we can conjugate all maximals around
> r := Random(G`RandomProcess);
> for H in l do
for>     time g, slp := SuzukiMaximalSubgroupsConjugacy(G, H, H^r);
for>     time print H^g eq H^r;
for> end for;
Time: 0.010
true
Time: 193.440
Time: 0.000
true
Time: 0.030
Time: 0.020
true
Time: 0.040
Time: 0.020
true
Time: 0.020
Time: 0.140
true
Time: 0.040
\end{verbatim}

\bibliographystyle{amsplain}
\bibliography{hb}

\end{document}
