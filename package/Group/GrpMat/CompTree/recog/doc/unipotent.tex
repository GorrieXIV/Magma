% $Id:: unipotent.tex 1330 2008-10-31 14:51:44Z jbaa004                      $:

\magnification 1200
\centerline{\bf Unipotent groups}
\medskip
The first Aschbacher category considered is reducible groups, and we have a unipotent kernel,
with the image a subdirect product of the diagonal blocks.

The generators that we store for unipotent groups will be normal subgroup generators, not subgroup generators.

The diagonal blocks define a rectangular block structure for the unipotent kernel, so this kernel  $U$ has a 
natural series, whose filtration quotients are modules for $G/U$.  Each rectangular block is a module for
a quotient of $G/U$ arising from two diagonal blocks, one acting on the left and the other on the right.

Given a unipotent node that is not rectangular, it will have as right child the top rectangular
block that it maps to.

The general process will be similar to the way that other nodes are dealt with, mandarins and all.

The basic driving process will be the trivial algorithm for spinning up a vector.  Given an $FG$-module
 $V$, and a vector $v$ in $V$, find a basis for the sub-module $U$ of $V$ generated by $v$.  We need to
 maintain SLP's for the basis elements.  More generally, given a basis $B$ for a submodule $W$ of
 $V$ we need to find a subset $C$ of $V$ such that $B\cup C$ is a basis for $W+\langle v^G\rangle$.
 
 By default the ground field is the prime field.  But if the two diagonal blocks $G_1$ and $G_2$
 acting on a rectangular block
 have centres that generate the fields $F_1$ and $F_2$ then we may regard the rectangular block
 as a vector space over the join of $F_1$ and $F_2$.
 
 We can now use our usual mechanism for getting random elements in a node corresponding to
 a unipotent node: if it is non-rectangular this gives rise to random elements in the right child,
 and hence, by the usual mechanism, in the left child.
 
 The shape of the tree defined by a unipotent node is described by observing that all right
 children are leaves.  We may have constant crises if we try to construct children from the
 generators that we have for the parent.  The following may work better.  To find a random element
 in a node we are processing take a random element at the top unipotent node and chase it down in the usual way.  It would then seem sensible to add this adjusted element to the top unipotent node.
 
 The essential difference between unipotent nodes and other types of node that suggest the use
 of normal subgroup generators are:
 
 1.  We have a very effective constructive membership test in unipotent nodes, namely echelonisation.
 Given an element in a node we can very efficiently construct it conjugates and test them for
 membership.
 
 2.  We may have need very many subgroup generators.
 
 \bye
