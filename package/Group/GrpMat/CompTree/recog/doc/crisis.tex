% $Id:: crisis.tex 1330 2008-10-31 14:51:44Z jbaa004                         $:

\magnification 1200
\centerline{\bf Crisis management}
A crisis occurs when we discover that we do not have a full generating set for a left child that
is not the node being processed, but is some ancestor of that node.

I assume that a result confirmed by the mandarins is in fact true.  The probability that this fails
is so very small that in theory one should simply start the whole computation again from the
beginning; but in practice one should suspect a bug.

The reason for having good crisis management is that we can then take greater risks.  That is
to say, we can keep the number of generators used down to a reasonably small number.
With unipotent leaves we may need many generators; I have some suggestions on how to
deal with this problem, dealt with elsewhere.

If the crisis management follows the lines suggested here the effect of a crisis will
be to leave the part of the tree already constructed almost completely unaltered.

When we process the left child of a node $N$, the right child of $N$ is mandarin checked.  
It follows that if we do not have a full generating set for $N$ the error has occurred when
we constructed an inadequate generating set for an ancestor of $N$ that is a left child.

The first step will be to consider in turn ancestors of $N$ that are left children, working upwards
towards the root, until we find a node $M$ where a new generator will give rise to an element in
$N$ that covers the error.  That is to say, an element that points to an error at $N$, probably
the same error that the mandarin noticed.

Say that a node that lies on the path from $M$ to $N$ a relevant node.

The fix will consist of adding a new generator to each relevant node.

If $P$ is a relevant node, the right child of $P$ has been processed and mandarin checked;
the left child of $P$ has not been constructed.

Let $P$ be a relevant node with a relevant child $Q$.

If $Q$ is the left child of $P$ then the new generator of $P$ will map to the identity in the
right child of $P$; in other words, it will already lie in the group defined at $Q$.  This means
that the right child of $P$, which has been checked, is now given a new generator that is
the identity, and may, for technical reasons,  have to be propagated downwards to the right until it disappears as the identity in a leaf.

Here is the simple algorithm that achieves this.

Locate the the node $M$.  This implies that we have found a new generator $g$ that we can
add to $M$ that gives rise to a useful element in $N$; that is to an element that points to the error.
We assume that this is the youngest ancestor of $N$ that has this property.

We now apply a procedure Fix.  Fix has as parameters a relevant node $Q$, and an extra
generator $h$ for the group at $Q$.  The relevant ancestors of $Q$ also have one extra generator
each.  These extra generators will be updated from time to time.  If $P$ is the relevant
parent of $Q$ (assuming that $Q$ is not $M$) then either $Q$ is the right child of $P$ and
the extra generator of $P$ maps to the extra generator of $Q$, or $Q$ is the left child of
$P$ and the extra generator for $P$ is the extra generator for $Q$.  The extra generators
come with the appropriate SLP's.

Fix terminates when the node parameter $Q$ is $N$.

if the relevant child of $Q$ is the right child $R$, take the image of the extra generator for
$Q$ to be the extra generator for $R$, and apply Fix to $R$;

if the relevant child of $Q$ is the left child $L$ then

if the extra generator of $Q$ lies in $L$ apply Fix to $L$ with this extra generator;

else write the image of the extra generator in $R$ as an SLP $w$ in the generators of $R$, and
apply AdjustExtraGenerators, with parameters $M$, $Q$ and $w$.  This changes the 
extra generators in the relevant nodes between $M$ and $Q$;

apply Fix with parameters $Q$ and the new extra generator for $Q$;

end else;

end Fix;

AdjustExtraGenerators divides the extra generator by $w$ evaluated on the generators of the 
appropriate node.  This is straightforward when going from right child to parent.
When going from left child to parent the generators of the left child are SLP's in the
generators of the parent, so at this stage the SLP $w$ is replaced by a longer SLP
on the generators of the parent.

Note that AdjustExtraGenerators multiplies the extra generators by elements of the
old generators of $M$.  Thus the property of not pointing to the problem in $N$
is preserved.  If $M$ has been incorrectly identified, in that a younger ancestor
of $N$ would suffice, there is a small risk that the extra generator returned by Fix
will in fact not fix.  If this happens one should simply start fixing the crisis again from
scratch, perhaps noting that one needs to find a younger node for $M$.

\bye
