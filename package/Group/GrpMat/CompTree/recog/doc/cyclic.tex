\magnification 1200
\def\GL{{\rm GL}}
\def\GF{{\rm GF}}
\def\SL{{\rm SL}}
\centerline{\bf Data structures for matrix recognition.  Cyclic groups}
\medskip
The critical issue is to determine when we can avoid the use of discrete logs in a large field.

There are two types of cyclic nodes; abstract and field.  The only ones where the cyclic group can
become large is in the field type.

The current strategy ensures that large cyclic groups only arise in nodes defined by taking determinants.
I don't think that this is an important point from a strategic point of view; but it simplifies a description
of what is going on.

If we are to construct a composition series we have to know what primes turn up as orders of composition factors, so integer factorisation has to be able to cope
with the order of the image of the determinant.  So now we have a list of bad primes, for which we are
unwilling or reluctant to carry out discrete log calculations.  

A small point.  We are concerned with large primes dividing $q-1$ (or $q^e-1$ for some $e$).  The
primes are not so large that we can't find them, but they are too big for comfortable discrete logs.
 Does it ever occur in practice that such a large
prime divides $q-1$ twice?  I shall assume that this does not happen to simplify the discussion.

A chief factor of order a power of a large prime $\ell$ lies in $O_{p,\ell}(G)$.  In other words, if we divide
out by $O_p(G)$, and then by $O_\ell(G/O_p(G))$, we are left with a group with no composition
factor of order  $\ell$.
So we can put all the chief factors of order a power of $\ell$ at the bottom of $G/O_p(G)$, where
they will define an elementary abelian group (I think); that is to say, we have a direct sum of
$F_\ell(G)$-modules.  Call this the $\ell$-layer.

The goal for which I am aiming is as follows.  Without the use of discrete logs we arrive at this position,
but where each homogeneous factor of the $\ell$-layer is a subspace of $F_\ell^k$ for some known $k$, but the dimension of the space (that is, the multiplicity of the composition factor) may not be known.

We can then carry out a limited number of discrete log calculations to remove these ambiguities.

The first step in processing a group is to divide out by $O_p(G)$, leaving a group that acts
completely reducibly, say on the direct sum of $k$ irreducible modules.  Some of these irreducibles
may be written over a larger field, perhaps with Galois automorphisms.  Suppose that some large prime $\ell$ arises as a determinant, so $\ell$ divides $q-1$, or $q^e-1$ in a larger field context.  If
$\ell$ divides $q-1$ then $\ell$ arises as a determinant order in, say, $k_0$ of the irreducible
modules.  Now we have a node of type elementary abelian, with parameters $\ell$ and $k_0$,
and data structure ordered $k_0$-tuples of field elements of order $\ell$.  We cannot determine
relations between the elements without resort to discrete logs.  But we will manage without, though
at some stage we need to resolve the problem of the rank of this elementary abelian group.

If $\ell$ divides $q^e-1$, where an irreducible factor has been written over $\GF(q^e)$, but does not
divide $q-1$ the situation is slightly different.  The Galois group may act on the $\ell$ factor.  Now if
there are more than one such irreducible factor for the given $\ell$ the critical, and easily decided,
question is whether the $\ell$ factors are isomorphic as modules.  For example, the Galois group
might be of order 2, inverting the $\ell$ cycle, and this might happen in two irreducible factors of
the module.  Now if the Galois groups act on the two $\ell$-cycles as the Klein 4-group the
$\ell$-cycles must be different (and are non-isomorphic as $G$-modules); but if the Galois group
acts as a single 2-cycle we have an ambiguity, and we have either one or two $\ell$-cycles
inverted by this $C_2$.

There is another situation where $\ell$-cycles may occur.  Suppose that we have a system of imprimitivity, giving rise, for example, to $\GL(r,q)\wr S$, for some transitive permutation group $S$ on
$s$ symbols.  The discrete log problem gives rise to a difficulty, in that we
don't know whether the base group modulo its derived subgroup is the full direct product of
$s$ copies of $C_\ell$, or is some subgroup of this.  This can be sorted out, I think, by
using the action of $S$.  For example if $s=2$ then either both copies of $C_\ell$ arise, and
you have the full wreath product, or you have one of the diagonal subgroups $\{(a,a):a\in C_\ell\}$ or 
$\{(a,a^{-1}):a\in C_\ell\}$, and it is easy to see which is the case.  In fact the case $s=2$ is
unrealistic, in that the determinant of the whole group will have been taken earlier, so when we
get down to the wreath product we will be in the $(a,a^{-1})$ case.  For larger $s$ there is more to be done.

In any case we have a node whose universal group is $C_\ell^k$, elements being represented by
$k$-tuples of field elements of order $\ell$, but given as polynomials in some primitive element
(additive form) rather than as powers of a primitive element (multiplicative form).  So we can perform
the black box operations of multiplication, division, and testing for the identity, but
explicit membership is slow.

At some point we need to determine the rank of this $\ell$-group.  Using tricks such as those
above may help; and if they prove that the rank is 1 (which may be obvious) we don't need
discrete logs at all.

The critical point is to avoid the use of discrete logs to construct the rest of the composition tree.
This reduces their use to a minimum.

We use explicit membership in a right node to construct random elements in the left node.  If we have
a node whose right child is a node of type elementary abelian, with
$\ell$ as the prime, we construct a random element of the left node by taking a random 
element of the derived group of the parent node, and multiplying it by the $\ell$-th power of
a random element of the parent node.  This can be used both for mandarins and workers.  It works
because the left child of the parent node is known not to map onto $C_\ell$.

I think that the following simple recipe works.

Whenever we are looking at a node that acts completely reducibly (no $O_p$), look at the determinant
of its restriction to the various irreducible blocks, increasing the field first where possible.  If any of these determinants involve a large prime $\ell$, map to a
node of type elementary abelian, with one parameter the number of blocks whose determinants 
involve $\ell$, and the other $\ell$ itself.  Now taking determinants and powering (to kill other primes)
gives the required homomorphism onto this node.

Now you can navigate round this node rapidly, and at some date determine (using discrete logs if
need be) what the actual rank is.

\bye