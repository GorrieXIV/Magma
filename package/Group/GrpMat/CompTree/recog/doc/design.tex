%******************************************************************************
%
%    design.tex  Composition Tree Design
%
%    File      : $HeadURL:: https://subversion.sfac.auckland.ac.nz/svn/prj_m#$:
%    Author    : Henrik B��rnhielm and Eamonn O'Brien
%    Dev start : 2008-10-23
%
%    Version   : $Revision:: 1493                                            $:
%    Date      : $Date:: 2008-11-26 11:22:36 +1100 (Wed, 26 Nov 2008)        $:
%    Last edit : $Author:: jbaa004                                           $:
%
%    $Id:: design.tex 1493 2008-11-26 00:22:36Z jbaa004                      $:
%
%******************************************************************************

\documentclass[twoside,a4paper,reqno,psamsfonts]{amsart}
%\usepackage[swedish,english]{babel} 
\usepackage[latin1]{inputenc} % svenska tecken skall tolkas
%\usepackage{amsmath}
%\usepackage{amsfonts} % for \mathbb
%\usepackage{amsthm}
\usepackage{amscd}
\usepackage{amsopn}
\usepackage{amstext}
\usepackage{amsxtra}
\usepackage{amssymb}
\usepackage{stmaryrd}
\usepackage{upref}
\usepackage{textcase} % For correct case of swedish chars
%\usepackage[algo2e,ruled,linesnumbered,algosection]{algorithm2e} % for typesetting algorithms
\usepackage{clrscode}
%\renewcommand{\Comment}{$\hspace*{-0.075em} //$ } % Use // for comments
\usepackage{bm} % for \boldsymbol
\usepackage{draftcopy}
\usepackage{graphicx}
\usepackage{setspace}
\frenchspacing

%\numberwithin{section}{chapter}
\numberwithin{equation}{section}
\numberwithin{figure}{section}
%\numberwithin{algorithm}{chapter}

\newcounter{algorithm}
%\numberwithin{algorithm}{section}
%\renewcommand{\thealgorithm}{\arabic{section}.\arabic{algorithm}}

\theoremstyle{plain}
\newtheorem{thm}{Theorem}[section]
\newtheorem{lem}[thm]{Lemma}
\newtheorem{cl}[thm]{Corollary}
\newtheorem{pr}[thm]{Proposition}
\newtheorem{axiom}[thm]{Axiom}
\newtheorem{conj}[thm]{Conjecture}

\theoremstyle{definition}
\newtheorem{deff}[thm]{Definition}

\theoremstyle{remark}
\newtheorem{note}[thm]{Note}
\newtheorem{rem}[thm]{Remark}

%\renewcommand{\thetheorem}{Theorem~\arabic{chapter}.\arabic{theorem}}
%\renewcommand{\thepr}{Proposition~\arabic{chapter}.\arabic{pr}}
%\renewcommand{\thedeff}{Definition~\arabic{chapter}.\arabic{deff}}
\providecommand{\abs}[1]{\left\lvert #1 \right\rvert}
\providecommand{\norm}[1]{\left\lVert #1 \right\rVert}
\providecommand{\ceil}[1]{\left\lceil #1 \right\rceil}
\providecommand{\floor}[1]{\lfloor #1 \rfloor}
\providecommand{\set}[1]{\left\lbrace #1 \right\rbrace}
\providecommand{\gen}[1]{\left\langle #1 \right\rangle}
%\providecommand{\ord}[1]{\operatorname{ord}( #1 )}
\providecommand{\Sym}[1]{\operatorname{Sym}( #1 )}
\renewcommand{\Pr}[1]{\operatorname{Pr}[ #1 ]}
%\renewcommand{\char}[1]{\operatorname{char}[ #1 ]}

\newcommand{\field}[1]{\mathbb{#1}}
\newcommand{\vect}[1]{\boldsymbol{\mathrm{#1}}}
\newcommand{\N}{\field{N}}
\newcommand{\Z}{\field{Z}}
\newcommand{\R}{\field{R}}
\newcommand{\Q}{\field{Q}}
\newcommand{\OO}{\field{O}}
\newcommand{\K}{\field{K}}
\newcommand{\A}{\field{A}}
\newcommand{\F}{\field{F}}
\newcommand{\PS}{\field{P}}
\newcommand{\GAP}{\textsf{GAP}}
\newcommand{\MAGMA}{\textsc{Magma}}

% for Cayley graphs
\newcommand{\C}{\mathcal{C}}
\newcommand{\OV}{\mathcal{O}}

\DeclareMathOperator{\lcm}{lcm}
\DeclareMathOperator{\sgd}{sgd}
\DeclareMathOperator{\mgm}{mgm}
\DeclareMathOperator{\sgn}{sgn}
\DeclareMathOperator{\GL}{GL}
\DeclareMathOperator{\PGL}{PGL}
\DeclareMathOperator{\GF}{GF}
\DeclareMathOperator{\IM}{Im}
\DeclareMathOperator{\RE}{Re}
\DeclareMathOperator{\I}{Id}
%\DeclareMathOperator{\OR}{O}
\DeclareMathOperator{\SL}{SL}
\DeclareMathOperator{\GO}{GO}
\DeclareMathOperator{\SO}{SO}
\DeclareMathOperator{\Sz}{Sz}
\DeclareMathOperator{\Sp}{Sp}
\DeclareMathOperator{\SU}{SU}
\DeclareMathOperator{\chr}{char}
\DeclareMathOperator{\Aut}{Aut}
\DeclareMathOperator{\Alt}{Alt}
\DeclareMathOperator{\PSL}{PSL}
\DeclareMathOperator{\PSp}{PSp}
\DeclareMathOperator{\PPSL}{(P)SL}
\DeclareMathOperator{\diag}{diag}
\DeclareMathOperator{\Tr}{Tr}
\DeclareMathOperator{\SLP}{\mathtt{SLP}}
\DeclareMathOperator{\G2}{{^2}G_2}
\DeclareMathOperator{\LargeRee}{{^2}F_4}
\DeclareMathOperator{\Ree}{Ree}
\DeclareMathOperator{\Gal}{Gal}
\DeclareMathOperator{\Norm}{N}
\DeclareMathOperator{\Cent}{C}
\DeclareMathOperator{\Zent}{Z}
\DeclareMathOperator{\EndR}{End}
\DeclareMathOperator{\Hom}{Hom}
\DeclareMathOperator{\Ker}{Ker}
\DeclareMathOperator{\cln}{{:}}
\DeclareMathOperator{\O2}{O_2}
\DeclareMathOperator{\Op}{O_p}
\DeclareMathOperator{\RP}{\bf{RP}}
\DeclareMathOperator{\NP}{\bf{NP}}
\DeclareMathOperator{\PP}{\bf{P}}
\DeclareMathOperator{\coRP}{\bf{co-RP}}
\DeclareMathOperator{\ZPP}{\bf{ZPP}}
\DeclareMathOperator{\Imm}{Im}
\DeclareMathOperator{\Mat}{Mat}
\DeclareMathOperator{\Dih}{D}


\newcommand{\OR}[1]{\operatorname{O} \bigl( #1 \bigr)}
%\newcommand{\cln}{\operatorname{O} ( #1 )}

\title{Composition Tree in MAGMA}

\author{Henrik B\"a\"arnhielm}
\address{Department of Mathematics \\ University of Auckland \\ Auckland \\ New Zealand}
\urladdr{http://www.math.auckland.ac.nz/\textasciitilde henrik/}
\email{henrik@math.auckland.ac.nz}

\author{Charles Leedham-Green}
\address{School of Mathematical Sciences \\ Queen Mary, University of London \\ Mile End Road \\ London \\ United Kingdom}
\urladdr{http://www.maths.qmul.ac.uk/\textasciitilde crlg/}
\email{c.r.leedham-green@qmul.ac.uk}

\author{Eamonn O'Brien}
\address{Department of Mathematics \\ University of Auckland \\ Auckland \\ New Zealand}
\urladdr{http://www.math.auckland.ac.nz/\textasciitilde obrien/}
\email{obrien@math.auckland.ac.nz}

\begin{document}

\begin{abstract}
Design description and documentation for $\MAGMA$ implementation of \texttt{CompositionTree}.
\end{abstract}

\maketitle

\section{Introduction}

This is a description of the implementation of the
\texttt{CompositionTree} $\MAGMA$ package. The motivation comes
from the \emph{matrix group recognition project} \cite{crlg01, MR2258009}, where the desire is
to compute the structure of a given matrix group. Permutation group
techniques (BSGS) can be used for matrix groups up to a certain degree
and finite field size, but beyond that something else is necessary.
This package is intended for that purpose. 

One would like to
fit into the \emph{Soluble Radical} method of \cite[Chapter 10]{hcgt},
since this enables the use of many algorithms that are implemented in
$\MAGMA$.  The method of \cite{MR2429998, statherthesis} achieves
this, but this package fit into an alternative and simpler approach.

\begin{enumerate}
\item Compute a \emph{composition tree} of the given matrix group,
  which is what this package does. The overall strategy for this is
  described in \cite{crlg01}, but new ideas from \cite{MR2289128} will
  be used. 
\item Compute a composition series of the group from the composition tree.
\item Rearrange the composition factors in order to obtain a chief series. 
\end{enumerate}

This approach has the advantage that the algorithm of \cite{crlg01} is
simpler than the one in \cite{MR2429998}, and therefore easier to
implement and optimise. The rearranging is also a simple algorithm,
and can use the composition tree as a black box with a well-defined
interface, without knowing how it was produced. The disadvantage
towards \cite{MR2429998} is that certain groups will require a lot of
rearranging\footnote{The typical example is imprimitive matrix groups
  like $\SL(2, 7) \wr \Sym{20}$.}.

A composition tree of a group can be seen as a data structure for
storing composition factors, which can be built up recursively. It can
also be seen as a data structure that facilitates constructive
membership testing (rewriting) in a group, using the algorithm
described in \cite{crlg01}. The ability to perform rewriting is the
only thing that is immediately available when a composition tree has
been constructed, and it is an essential feature, which is also
necessary in the recursive construction algorithm. It depends on the
availability of rewriting algorithms in the finite simple groups, a
dependency which is only almost satisfied. To efficiently obtain
composition series we also need constructive recognition algorithms for
finite simple groups. In the cases where implementations of these are
not available, we have to rely on BSGS computations and on
\cite{ryba_trick}.

A \emph{rewriting} algorithm for a group $G$ is an algorithm for the
\emph{constructive membership problem} in $G$: given $g \in U
\geqslant G = \gen{X}$, decide whether or not $g \in G$, and if so
express $g$ as a straight line program in $X$. The first component is
the memberhip problem, and sometimes only the second component is
called rewriting. The \emph{constructive recognition problem} is as
follows: construct an isomorphism $\varphi$ from $G$ to a
\emph{standard copy} $H$ of $G$ such that $\varphi(g)$ can be computed
efficiently for every $g \in G$. Such an isomorphism is called
\emph{effective}. Sometimes one also demands that $\varphi^{-1}$ is
also effective.

Constructing the composition factors from the tree requires more work,
occasionally much more, typically when the tree includes leaves where
we have to compute BSGS. This is the reason that constructing the tree
and the series are separate operations. One may wish to only have the
tree, if one is only interested in rewriting and not in the
composition series.

The overall strategy for building the composition tree is as follows.

\begin{enumerate}
\item Given a group $G$, construct an effective homomorphism $\phi : G
  \to I$, for some group $I$, or notice that $G$ is simple. If $G$ is
  simple, it is a leaf. Assume henceforth that $G$ is not simple.
\item Construct a composition tree for $I$.
\item Construct generators for $K := \Ker(\phi)$.
\item Construct a composition tree for $K$.
\item Put together the composition trees for $I$ and $K$ into a tree for $G$.
\end{enumerate}

Here, the ability of rewriting in $I$ is necessary in order to construct $K$.

The overall strategy above does not by itself assume that the input is a matrix
group, but this assumption is needed in the first step. From
\cite{crlg01}, we can use Aschbacher's theorem \cite{aschbacher84} for
matrix groups. This requires algorithms for deciding if $G$ lies in a
certain Aschbacher class, and to construct the corresponding $\phi$.
Such algorithms exist and are implemented in $\MAGMA$.

As a consequence, in the matrix group context a leaf is not
necessarily simple. A matrix leaf is either a classical group in
natural representation or a central extension of a non-abelian finite
simple group in matrix representation, \emph{i.e.} simple modulo
scalars. Leaves can also be cyclic, not necessarily of prime order,
but these are not stored as matrix groups.

One can proceed in a similar way for permutation groups, using the
O'Nan-Scott theorem instead of the Aschbacher theorem in order to
implement the first step. However, in the permutation group context
there is already an extensive library of algorithms which only require
a BSGS. For a small base permutation group one can easily find a BSGS,
so our goal in the permutation group context is only to apply
reductions to small base groups and then use the existing permutation
group machinery.

\section{Interface for factor rearranging}
\label{section:rearrange_interface}
The algorithm that rearranges the composition factors requires the
ability to construct a composition series together with a number of
homomorphisms. For a group $G$, it requires the following.
\begin{itemize}
\item A composition series $\gen{1} = G_0 < G_1 < \dotsm < G_k = G$, hence generating sets for each $G_i$. 
\item For each $i > 0$, an effective homomorphism $\varphi_i : G_i \to S_i$,
  where $S_i$ is a \emph{standard copy} of $G_i / G_{i - 1}$. The
  standard copy should be chosen so that computations inside $S_i$ are
  as easy as possible. The kernel of $\varphi_i$ should be $G_{i -
    1}$. However, it is allowed that $S_i$ is not simple, but a
  central extension of a finite simple group. In this case,
  $\Ker(\varphi_i) = (G_{i - 1} . Z) / Z$ where $Z =
  \varphi_i^{-1}(\Zent(S_i))$. The typical example where this is
  necessary is when $G_i / G_{i - 1} \cong \PSL(d, q)$, then $S_i =
  \SL(d, q)$, because it is too expensive to explicitly write down $\PSL(d, q)$ (a permutation group) when $d$ or $q$ are large.
\item For each $i > 0$, the effective inverse $\varphi_i^{-1}$.
\item For each $i > 0$, a rewriting algorithm for $S_i$.
\end{itemize}

An additional requirement is an algorithm that, given $G$ and $g \in
G$, returns the minimal $i$ such that $g \in G_i$. Given the above
homomorphisms, this algorithm is as in Algorithm
\ref{alg:factor_number}. It only relies on the observation that for
any $j$, if $\varphi_j(g) = 1$ then $g \in \Ker(\varphi_j) = G_{j - 1}$ and
therefore the required minimal $i$ is less than $j$.

\begin{figure}[hb!]
\begin{codebox} 
\refstepcounter{algorithm}
\label{alg:factor_number}
\Procname{\kw{Algorithm} \ref{alg:factor_number}: $\proc{FactorNumber}(g)$}
\li \For $i := k - 1$ \Downto $1$ \By $-1$
\zi \Do
\li $h := \varphi_i(g)$
\li \If $S_i$ is simple
\zi \Then \If $h \neq 1$
\li \Then \Return{$i + 1$}
    \End
\li \Else \If $h \notin \Zent(S_i)$ 
\li \Then \Return{$i + 1$}
    \End
    \End
    \End
\zi \kw{end}
\li \Return{$1$}
\end{codebox}
\end{figure}

\section{Data structure}
A composition tree is stored as a full binary tree. Each node
represents a group and has various data attached to it. The right
child represents the image of a reduction and the left child
represents the kernel. 

If $G$ is the node group, the relationship is that the group $I$ of
the right child should be the image of $G$ under a homomorphism
$\phi$, and the group $K$ of left child is then $\Ker(\phi)
\trianglelefteqslant G$. However, we need to generalise this slightly.
It is allowed that the right child group is only the image of the
homomorphism after dividing it by a subgroup of the group generated by
its scalar flag. More details about this in Section
\ref{section:scalars}. We also allow that the kernel is not strictly a
subgroup, but isomorphic to a subgroup. The isomorphisms must be
effective and should be constructed at the same time as $\phi$. This
abstracts away the details of the mappings between the node and its
children from the rest of the algorithm.

This generalisation allows us to store the kernel in a different
representation, which can be more efficient. For instance, if $\phi$
is a tensor decomposition of a matrix group, then we want to store
both the image and the kernel as matrix groups of smaller dimension.

Here is a list of the important data elements stored in a node, after
the composition tree rooted at the node is constructed.

\begin{itemize}
\item A list of input generators, generating the group $G$ of the
  node. If the node is a right child, then these should be in
  bijection with the input generators of the parent.
\item An instance of the product replacement algorithm \cite{lg95, lgsm02, MR1829489, Pak00} on the input generators.
\item A list of reduction algorithms that can be used to find a
  reduction homomorphism, in a given order. If the group is a matrix
  group, this is the list of Aschbacher reduction algorithms, but the
  order may need to change at runtime. For example, if a matrix group is
  tensor induced, then the kernel is a tensor product, so the order of
  the reduction algorithms in the kernel can be changed so that the
  tensor decomposition algorithm is first.
\item A list of \emph{nice} generators, also generating $G$. In an
  internal node, these consist of the nice generators of the kernel
  and image. In a leaf the nice generators are not specified in
  general, but for example in a classical group, they are the standard
  generators of the leaf group. The idea is to replace the input
  generators with a smaller generating set, which allows for having
  shorter $\SLP$s in constructive membership testing.
\item A list of $\SLP$s of the nice generators, in the input generators.
\item A rewriting algorithm, that expresses any element of $G$ as an $\SLP$ in the nice generators.
\item A list of elements used to verify the the subtree rooted at the
  node. This is done by rewriting the elements using the constructed
  rewriting algorithm, thus verifying that the elements lie in the
  node group. These elements, that serve the purpose of quality control,
  are called the \emph{mandarins} (after the government officials in
  Imperial China.)
\item A list of $\SLP$s of the mandarins, in the nice generators.
\item A list of $\SLP$s of relators for a presentation on the nice generators.
\item A list of lists of $\SLP$s of generators for each composition factor of $G$, in the nice generators.
\item The three lists of homomorphisms described in Section \ref{section:rearrange_interface}.
\item The scalar flag for $G$. If $G$ is a permutation group this is always the identity. If $G \leqslant \GL(d, q)$ it is a scalar matrix $\alpha \I_d$ where $\alpha \in \F_q^{\times}$.
\end{itemize}

In the case that the node is not a leaf, some additional data is stored.
\begin{itemize}
\item A list of $\SLP$s of the generators of the left child, the kernel, in the input generators of the node.
\item The reduction homomorphism $\phi$.
\end{itemize}

In the case that the node is a leaf, there are also some additional data.

\begin{itemize}
\item The name of the leaf group, from \cite{general_recognition}.
\item Generators of the standard copy $S$. This is typically the
  corresponding finite simple group in a small degree representation.
  Hence it is not necessarily isomorphic to the leaf group, in the
  case where this is a central extension of the simple group.
\item An effective isomorphism $\alpha : G \to S$.
\item An effective isomorphism $\beta : S \to G$.
\end{itemize}

To be precise, $\alpha$ and $\beta$ are allowed to be isomorphisms
modulo scalars. Hence we require that if $G \cong Z.S$ where $Z =
\Zent(G)$ is a cyclic group, then $\beta(\alpha(g)) \in Z$ for every
$g \in G$. If $G$ is a matrix group, then $Z$ consists of scalar
matrices, because a matrix leaf is absolutely irreducible. If $G$ is a
permutation group leaf which is a central extension of a simple group,
then it is in fact simple, by the O'Nan-Scott theorem, so $Z =
\gen{1}$ in this case.

\section{General algorithm}
We now describe aspects of the general composition tree algorithm.

\subsection{Main recursion}
The main algorithm proceeds as follows.
\begin{enumerate}
\item Try to find a reduction homomorphism $\phi : G \to I$. Hence we try each reduction algorithm in turn, until one succeeds, see Sections \ref{section:matrix_reductions} and \ref{section:perm_reductions}.
\item If no reduction is found, the node is marked as a leaf and the main recursion is finished, see Section \ref{section:leaf_algorithm}. 
\item Setup the data structure for the right child, and recursively find a composition tree.
\item Find a probable generating set for $K$, see Section \ref{section:kernel_generation}.
\item Recursively find a composition tree for $K$. This may provoke a crisis, see Section \ref{section:kernel_generation}.
\item Setup the nice generators and the rewriting algorithm for the node.
\item Verify the rewriting using the mandarins. This may provoke a crisis.
\item Possibly construct a presentation from $I$ and $K$ and see if
  $G$ satisfies it, see Section \ref{section:verification}. This may provoke a crisis.
\end{enumerate}
In the case of a crisis, if the node is safe we add more generators to $K$
and proceed from that step, otherwise we send the crisis to the parent.

\subsection{Kernel generation}
\label{section:kernel_generation}

We use two general methods for constructing a probable generating set
for the kernel $K$. The first is the method described in
\cite{crlg01}:
\begin{enumerate}
\item Construct a set $Y = \set{g_1, \dotsc, g_n}$ of random elements of $G$ using the random process in the node.
\item Obtain $\SLP$s for $\set{\phi(g) : g \in Y} \subseteq I$, using
  the rewriting algorithm in the right child.
\item We obtain the $\SLP$s in the input generators of $I$, which are
  in bijection with the input generators of $G$. Hence we can evaluate
  the $\SLP$s in $G$ and obtain a set $\bar{Y} = \set{\bar{g}_1, \dotsc, \bar{g}_n}$ corresponding to $Y$.
\item Now $g_i / \bar{g}_i \in K$ for each $i$, and this is a random
  element of $K$.
\end{enumerate}

This method has no provable probability of success in general, and is
only a heuristic. For a given group it is a Monte Carlo algorithm,
although with unknown success probability, and as a heuristic it is
very practical.

The other method assumes that a presentation of $I \cong \gen{Y \mid
  R}$ is available. In our case, $Y$ is the nice generating set of
$I$, but we can easily rewrite the presentation on the input
generators.
\begin{enumerate}
\item Let $G = \gen{X}$. Evaluate the relators $R$ on $X$. This relies on the
  bijection between the input generators of $I$ and $X$, and produces
  $Z_1 \subseteq K$.
\item Map $X$ to $I$, obtain $\SLP$s and divide, as in the previous method, to obtain $Z_2 \subseteq K$.
\item Now $Z = Z_1 \cup Z_2$ is a normal generating set for $K$. In other words, $K$ is the normal closure of $\gen{Z}$. Note that $\abs{Z} = \abs{R} + \abs{X}$.
\item Use the algorithm in \cite{MR1235795} to construct probable
  generators of the normal closure of $\gen{Z}$. This is a Monte Carlo
  algorithm, and the number of generators it constructs depends on the
  given error probability and the length of the longest subgroup chain
  in $G$, usually denoted $l_G$.
\end{enumerate}

Note that there are no practical upper bounds on $l_G$. If $G
\leqslant \GL(d, q)$ then $l_G \leqslant d^2$, and this bound is
optimal since it is achieved by $p$-groups. If $\mathrm{O}_p(G) =
\gen{1}$ then $l_G \leqslant d$, which is better but still not useful
in practice, and this bound is also optimal. If $G \leqslant \Sym{n}$
then $l_G \leqslant n$, again an optimal upper bound. In practice, we
use a very small value of $l_G$.

We prefer the second method of generating the kernel, whenever it can
be used. It depends upon our ability to obtain presentations of the
leaves, because we can construct a presentation of a node if both its
children have presentations. A drawback is that the number of relators
can be very large, and hence the number of kernel generators will be
large.

Note that both these methods of finding the kernel are Monte Carlo, so
there is a possibility that we fail to find the full kernel. This has
consequences for the design of the composition tree algorithm. It means that
constructive membership testing is insecure, if it reports that an
element does not lie in a group corresponding to some node, this may
mean that some left child on the path from the node to the root was
not calculated correctly. This situation is called a \emph{crisis},
and if it occurs we must discard a part of the tree and recompute it.
The mandarins are used to test constructive membership testing, to try
to provoke a crisis.

If there are several kernel computations on the path to the root, we
have no way of knowing which one of them that is incorrect. We
therefore introduce the notion of a \emph{safe} node.
\begin{itemize}
\item The root node is safe.
\item A right child is safe if and only if its parent is safe.
\item A left child is never safe unless it is calculated using some
  special method that is guaranteed to give the full kernel.
\end{itemize}
Now if we get into a crisis during rewriting in some node, we must
backtrack along the path from the node to the root, until we reach a
safe node. We must discard the whole left subtree of this node, except
that we store the generators of its left child. Then we can add more
generators of this kernel and try again.

\subsection{Verification}
\label{section:verification}

The composition tree algorithm is inherently probabilistic and there
are many computations along the way which return results that may be
wrong with some positive probability. In effect it is a Monte Carlo
algorithm, although the success probability is unknown, and it is not
clear that there exists a constant upper bound on this probability if
the input is an arbitrary matrix group over a finite field.

Therefore there is a positive probability that the composition tree is
only a composition tree for a subgroup of the input group.

To verify the composition tree, and hence eventually turn it a Las
Vegas algorithm, we calculate a presentation from the composition tree
and see if this is a presentation of the input group. In other words,
we test if all the relators, which must be relators for a subgroup,
are also relators in the input group.

There are two options, either we verify during construction the
composition tree, or afterwards. In the first case, when the
composition trees of the right and left children have been
constructed, they are immediately verified before putting them
together.

Presentations can be constructed recursively, given presentations of
the image and kernel. For a leaf, we require that the leaf algorithm
constructs a presentation on the nice generators. This is possible in
theory for all finite simple groups except the small Ree groups \cite{MR2393425, pres2,
  bray_sz_presentation, alt_sym_short_pres}, but not always possible in practice for every
type of leaf, in which case we cannot construct a presentation for the
composition tree.

To construct a presentation recursively for $G$, assume that $I \cong
\gen{X_I \mid R_I}$ and $K \cong \gen{X_K \mid R_K}$ are presentations
for the image and kernel, where $X_I$ and $X_K$ are the nice
generators and $R_I$ and $R_K$ are $\SLP$s of the relators. Recall
that the nice generators of $G$ is $X_G = X_K \cup \phi^{-1}(X_I)$,
where $\phi^{-1}(X_I)$ is calculated by evaluating the $\SLP$s of the
nice generators of $I$ on the input generators of $G$. This relies on
the bijection between the input generators of $G$ and $I$. Now proceed
as follows.
\begin{enumerate}
\item Rewrite $R_I$ on $X_G$ to obtain $R_I^G = \set{r_1^G, \dotsc,
    r_n^G}$ and evalute them on $X_G$ to obtain elements $Y_I \subseteq
  G$.
\item Since $R_I$ are relators of $I$, $Y_I \subseteq K$, so obtain $\SLP$s of $Y_I$ in $X_K$ and rewrite them on $X_G$, to obtain $R_I^K = \set{r_1^K, \dotsc, r_n^K}$. 
\item Let $Z_I = \set{r_i^G / r_i^K : i = 1, \dotsc, n}$.
\item Rewrite $R_K$ on $X_G$ to obtain $Z_K$.
\item Let $R_K^G = \set{s_1^G, \dotsc, s_m^G}$ be the $\SLP$s in $X_G$
  corresponding to \[\set{x^y : x \in X_K, y \in \phi^{-1}(X_I)}.\]
  Evaluate them on $X_G$ to obtain elements $C \subseteq G$.
\item Since $K$ is normal in $G$, $C \subseteq K$ and we can obtain $\SLP$s of $C$ in $X_K$. Rewrite these on $X_G$ to obtain $R_K^K = \set{s_1^K, \dotsc, s_m^K}$.
\item Let $Z_C = \set{s_i^G / s_i^K : i = 1, \dotsc, m}$.
\item Now $G \cong \gen{X_G \mid Z_I \cup Z_K \cup Z_C}$ is a presentation.
\end{enumerate}

Note that the number of relators in the presentation for $G$ is
$\abs{R_K} + \abs{R_I} + \abs{X_K} \abs{X_I}$, which can grow large
with the height of the tree.

\section{Reduction database for matrix groups}
\label{section:matrix_reductions}

The basic use of the Aschbacher reductions is already described in
\cite{crlg01}. Here we describe our use of these. The order in which they are described here is the standard order in which they are tried at runtime.

\subsection{Unipotent}
A group $G \leqslant \GL(d, p^e)$ is unipotent if and only if every $g
\in G$ has order $p^k \leqslant d$. Moreover, $G$ is unipotent if and
only if all composition factors of the module of $G$ have dimension
$1$, and $G$ acts trivially on each such factor.

We first employ a quick negative test to rule out most groups that are
not unipotent: if some of the given generators of $G$ do not have
order a power of $p$, then $G$ is not unipotent. If this test does not
rule out $G$, then we use the MeatAxe to decompose the module of $G$
and test if the composition factors behave as above.

If the group is unipotent, then MeatAxe provides a change of basis
matrix $c$ such that $G^c$ is lower triangular. Each subdiagonal of
$G^c$ is then a vector space over $\F_p$ and hence an elementary
abelian $p$-group, and the projection of $G^c$ to each such group is a
homomorphism. Hence we obtain a reduction to the first non-zero
subdiagonal.

The image is elementary abelian and treated as PC-group, so it is a
leaf. Hence we can readily obtain a presentation and therefore obtain
the kernel using the presentation method.

\subsection{MeatAxe reductions}
There are several reductions coming from the MeatAxe \cite{Parker84, meataxe, better_meataxe}. Let $M$ be
the module of $G \leqslant \GL(d, q)$. We first find the
indecomposable summands of $M$, so $M \cong M_1 \oplus \dotsb \oplus
M_k$. Then we find a composition series of each $M_i$, so $M_i =
M_{i,n_i} > M_{i, n_i - 1} > \dotsb > M_{i, 1} > M_{i, 0} = 1$. This
also provides a change of basis matrix $c_i \in \GL(M_i)$ that
exhibits this series. Then $c = \bigoplus_{i = 1}^k c_i$ (diagonal
join) exhibits the direct sum decomposition of $M$ as well as the
composition series of each $M_i$. The corresponding composition series
of $M$ is
\begin{equation}
1 = M_{1, 0} < M_{1, 1} < \dotsb < M_{1, n_1} < M_1 \oplus M_{2, 1} < \dotsb < M_1 \oplus M_2 < \dotsb < M
\end{equation}

Now $G^{c^{-1}}$ is both block lower triangular, corresponding to the
composition factors of $M$, and block diagonal, coresponding to the
direct summands of $M$. We obtain a homomorphism by projecting onto
the diagonal blocks corresponding to the composition factors. The
kernel of this is $\mathrm{O}_p(G)$. Since this is the first reduction
in the composition tree, unless $G$ is unipotent from the start, the
composition tree has the property that $\mathrm{O}_p(G)$ is the first
kernel. Therefore the composition factors coming from
$\mathrm{O}_p(G)$ will be at the bottom of the composition series
constructed from the tree. This is a useful, because it avoids later
rearranging of the (potentially large number of) composition factors
of $\mathrm{O}_p(G)$.

The group of the right child is now contained in $\GL(M_1) \times
\dotsm \times \GL(M_k)$, embedded in $\GL(d, q)$. From this node we
therefore obtain a homomorphism onto the first non-trivial summand
block. The right child of this reduction is
contained in $\GL(M_{1, 1}).\GL(M_{1, 2}). \dotsc . \GL(1, n_1)$,
embedded in $\GL(M_1)$, so from this node we obtain a homomorphism onto
the first non-trivial composition factor block of $M_1$.

Continuing in this way, we obtain reductions to the groups acting on
the composition factors of $M$. The reason for going via the summands
is to ensure that the matrices are as sparse as possible, and to
retain information on how $G$ acts on $M$. Moreover, we know that the
reduction to the last factor inside each summand must be an
isomorphism, because the kernel is trivial. If we had mapped to the
composition factors immediately we would have lost this information
and would had to calculate the kernel.

\subsection{Absolute reducibility}
Groups that act irreducibly but not absolutely irreducibly are part of
the \emph{semilinear} Aschbacher class. However, here we consider
these separately, because we have a much faster reduction homomorphism
in this special case. An irreducible group $G \leqslant \GL(d, q)$ is
absolutely reducible if is reducible when embedded into $\GL(d, q^e)$
for some $e > 1$. The smallest such $e$ gives the \emph{splitting
  field} $\F_{q^e}$ for $G$. Then $G$ has a faithful representation $G
\to \GL(d/e, \F_{q^e})$, so we obtain an isomorphism. This isomorphism
is in the $\MAGMA$ kernel and hence easy to calculate.

\subsection{Semilinearity}
For the general semilinear case we use part of \cite{MR1407872}, known
as SMASH. In this case there exists $N \triangleleft G = \gen{X} \leqslant
\GL(d, q)$ that is not absolutely irreducible, so it has a splitting
field $\F_{q^e}$. Moreover, there exists a non-scalar $C \in \Zent(N)$
such that for every $g \in G$ there exists $1 \leqslant i = i(g)
\leqslant e$ such that $Cg = gC^{q^i}$. The SMASH algorithm provides
$e$ and $C$, and we then obtain a homomorphism $G \to \Cent_e$,
given by $g \mapsto i(g)$.

In this case we can also immediately obtain generators for the kernel
$K$. Let the image $I = \gen{Y} \leqslant \C_e$, where $\abs{Y} =
\abs{X}$. Let $X = \set{x_1, \dotsc, x_m}$ and $Y = \set{y_1, \dotsc, y_m}$. 
\begin{enumerate}
\item Use the extended Euclidean
algorithm to obtain $n = \gcd(y_1, \dotsc, y_m, \abs{I})$ and an
expression $n = a_0 \abs{I} + \sum_{i = 1}^m a_i y_i$.
\item Let $x = \prod_{i = 1}^m x_i^{a_i}$ and $h_i = x_i x^{-y_i/n}$ for $i = 1, \dotsc, m$. Then $\set{h_1, \dotsc, h_m}$ is a normal generating set for $K$.
\item Let $k_0 = x^{\abs{I} / n}$ and $k_{i, j} = h_i^{x^j}$ for $i = 1, \dotsc, m$ and $j = 1, \dotsc, e$. 
\item Now $K = \gen{\set{k_0} \cup \set{k_{i, j} : 1 \leqslant i \leqslant m, 1 \leqslant j \leqslant e}}$.
\end{enumerate}

\subsection{Imprimitivity}
A matrix group $G \leqslant \GL(d, q)$ is imprimitive if it permutes a
non-trivial direct sum decomposition of $V = \F_q^d$. Hence $V \cong
V_1 \oplus \dotsm V_k$, where all $V_i$ are isomorphic and $G$
permutes these. This provides a homomorphism $G \to \Sym{k}$. The
SMASH algorithm is used to construct this homomorphism.

The kernel consists of those elements that preserves the decomposition of $V$,
and hence it is reducible. The SMASH algorithm provides a change of
basis matrix such that the kernel is block diagonal, and hence we can
immediately go to one of the MeatAxe reductions above, that map to the
direct summands. However, we find generator for the kernel we have to
use the general methods.

\subsection{Tensor products}
A matrix group $G \leqslant \GL(d, q)$ lie in this Aschbacher class if
its natural module $V = \F_q^d$ has the structure of a tensor product
$V \cong U \otimes W$, and $G$ respects this tensor product. It
follows that $G \cong H_1 \circ H_2$ where $H_1 \leqslant \GL(U)$ and
$H_2 \leqslant \GL(W)$. 

We use \cite{tensorprodalg}, with implementation by Eamonn O'Brien, to
obtain a change of basis matrix $c \in \GL(d, q)$ such that $G^c$ is
an explicit Kronecker product. It is straightforward to obtain the
factors of the Kronecker product, and hence we obtain a homomorphism
$G \to H_1$. This homomorphism is in fact only a homomorphism modulo
scalars, see Section \ref{section:scalars}.

\subsection{Tensor induction}
A matrix group $G \leqslant \GL(d, q)$ is tensor induced if its
natural module $V = \F_q^d$ has the structure of a tensor product $V
\cong U_1 \otimes U_2 \otimes \dotsm \otimes U_k$, where all $U_i$ are
isomorphic and $G$ permutes the tensor factors. This defines a
homomorphism $G \to \Sym{k}$.

We use the algorithm of \cite{MR1925006}, with implementation by
Eamonn O'Brien, to compute this homomorphism. 

We have no special method of obtaining the kernel in this case, but we
know that the kernel consists of those elements that preserves the tensor
product of $V$, and hence it falls in the previous Aschbacher class.
Hence we can immediately go to that reduction when processing the
kernel.

\subsection{Extra-special normalisers}
A group $G \leqslant \GL(d, q)$ belongs to this Aschbacher class if it
normalises an $r$-group $R$ of order $r^{2m + 1}$ or $2^{2m + 2}$,
where $r$ is prime, $r^m = d$ and $r \mid q - 1$. If $r > 2$ then $R$
is extraspecial, and if $r = 2$ then $R$ is either extraspecial or of
symplectic type, that is, a central product of an extraspecial
$2$-group and a cyclic group of order $4$.

If $m = 1$ then the algorithm of \cite{MR2142090}, with implementation
by Alice Niemeyer and Eamonn O'Brien, constructs a homomorphism $G \to
\GL(2, r)$. If $m > 1$ then the algorithm of \cite{MR2257997}, with
implementation by Henrik B��rnhielm, constructs a homomorphism $G \to
\GL(2n, r)$ or $G \to \Sym{r^n}$, where $1 \leqslant n \leqslant m$.

The latter algorithm also provides the kernel of the homomorphism, but
in the former case we have no special method of obtaining the kernel.

\subsection{Smaller fields}
In this case, $G \leqslant \GL(d, \F_q)$ is conjugate to $H \leqslant
\GL(d, \F_s) Z$ where $Z = \Zent(\GL(d, \F_q))$ and $\F_s < \F_q$. If
$H \leqslant \GL(d, \F_s)$ then this conjugation is an isomorphism $G
\to H$, so there is no kernel. In the general case, we obtain a
homomorphism $G \to H \cap Z$, so the image is a cyclic group. In
fact, this is a homomorphism modulo scalars, see Section
\ref{section:scalars}. The kernel is $H \cap \GL(d, \F_s)$, but we
have no special method of obtaining this kernel.

We use \cite{smallerfield}, with implementation by Eamonn O'Brien, to
compute the change of basis and to compute the homomorphism.

\subsection{C8}
The Aschbacher class C8 consists of the normalisers of classical
groups in natural representation. These are therefore not leaves,
but classical groups in natural representation are leaves. We use
three reductions to remove the decorations and obtain the classical
groups. All involves homomorphisms to cyclic groups. Unfortunately
this adds the requirement to calculate discrete logarithms,
potentially in the whole of $\F_q^{\times}$, since constructive
membership testing in a cyclic group is the discrete logarithm
problem.

To test if $G$ contains a classical group in natural representation,
we use \cite{MR1444141, MR1625479, MR1717416} with implementation by
Alice Niemeyer. If so, the node is marked as a leaf.

In fact, none of these reductions to cyclic groups are restricted to
the Aschbacher C8 class. They only rely on the existence of elements
with non-trivial determinants, and on a preserved classical form, but
a group may have these properties without being in the C8 class. 

Since all these reductions have cyclic images, which are leaves, we
can readily obtain presentations of the images and hence calculate the
kernels using the presentation method.

\subsubsection{Determinant}
The determinant map for $G \leqslant \GL(d, q)$ is $g \mapsto \det(g)
\in \F_q^{\times}$, so the image is a cyclic group. 

\subsubsection{Form action}
If $G \leqslant \GL(d, q)$ normalises a classical group $H$ which is
not linear, then $H$ preserves a classical form $J$. Hence $g J
\bar{g}^T = J$ for every $g \in H$, where $\bar{g} = g$ unless $H$ is
a unitary group, in which case $\bar{g} = g^{\sqrt{q}}$ ($q$ is a square
in this case). Then elements of $G$ preserves $J$ up to a scalar, so
for each $g \in G$ there exists $\lambda_g \in \F_q^{\times}$ such
that $g J \bar{g}^T = \lambda_g J$. Hence we obtain a homomorphism $g
\mapsto \lambda_g$, again with the image being a cyclic group.

To determine if the group preserves a classical form up to a scalar,
one can use the MeatAxe, as described in \cite[Section ?]{hcgt}. This
is implemented in MAGMA by Derek Holt.

\subsubsection{Spinor norm}
If $G \leqslant \GL(d, q)$ is an orthogonal group, then the previous
C8 reductions make sure that $G \leqslant \SO(d, q)$. But $\SO(d, q)$
is not perfect, and we again obtain a homomorphism $G \to \Cent_2$
where $g \in G$ maps to its spinor norm. The kernel of this is
$\Omega(d, q)$ which is simple modulo scalars.

To calculate spinor norms, we use \cite[Theorem 5.6]{spinornorms08}, with
implementation by Derek Holt.

\subsection{C9}
The Aschbacher class C9 consists of groups that are almost simple
modulo scalars. Hence $G \leqslant \GL(d, q)$ is a C9 group if $S
\leqslant G / Z \leqslant \Aut(S)$, where $Z = \Zent(G)$. It is
well-known that the automorphism group of a finite simple group is a
small extension of the simple group. In other words, $G \cong Z.S.E$
where $Z$ consists of scalar matrices and $E$ is a small soluble
group, usually cyclic. Because of our determinant reduction, $\abs{Z}
\mid \gcd(d, q - 1)$.

For groups in this class, we have to decide which finite simple group
they contain and we have to split off the decorations on top in order
to obtain a central extension of the simple group, which we treat as a
leaf. 

Here is an overview of our C9 method.

\begin{enumerate}
\item Calculate the \emph{stable derivative} $D = G^{(\infty)}$ of
  $G$. This is done by computing commutators, taking normal closures
  and testing for perfectness, see \cite{MR2429998} for more details.
  By a well-known result, we have to compute at most $3$ derived
  groups to obtain the stable derivative if $G$ is almost simple. Then
  $D$ is simple modulo scalars, but now the scalars are restricted to
  those in the Schur multiplier of $G$.
\item Try to name $D$. This is done in several steps, and if anyone of
  them succeeds we are done. It is possible that all fail, in which
  case $D$ is probably not a C9 group.
\begin{enumerate}
\item Use a heuristic implemented by Derek Holt to test if $D$ may be
  $\Alt(n)$ or $2{.}\Alt(n)$ (this requires knowledge of the centre),
  and try to find $n$. This is done more efficiently if we know
  $\Zent(D)$, so we first calculate it as in Section
  \ref{section:c9_centre}. If the heuristic is successful, use the
  constructive recognition algorithm \cite{MR1743388}, as
  verification. We use the implementation by Derek Holt.
\item Find a batch of random elements of $D$ and look at element
  orders to see if $D$ is a sporadic group, and find its name. If this
  appears to be the case, try to find projective standard generators
  of the sporadic group, as verification. The standard generators are
  given in the ATLAS \cite{atlas_www}.
\item Find the characteristic of $D$ using \cite{blackbox_char}. This was implemented by Eamonn O'Brien.
\item Find the name of $D$ and its defining field using
  \cite{general_recognition} (this requires that the characteristic is
  known). This was implemented by Eamonn O'Brien and Gunter Malle.
\end{enumerate}
\item Now we want to construct a homomorphism $G \to E$. This also
  uses the same general method as in \cite{MR2429998, statherthesis}.
  We try to find a complete set of coset representatives of $Z.S =
  \gen{Z, D}$ in $G$, and then construct the permutation
  representation of $G$ on these cosets. We employ a few observations
  that help us with this.
\begin{itemize}
\item If $D$ is named as a sporadic, then $G$ is at most the full
  automorphism group of this sporadic. We test this by trying to find
  projective standard generators (we calculate projective/central orders
  rather than precise orders of elements) of the automorphim group of
  the sporadic, again using black box algorithms from
  \cite{atlas_www}. If this succeeds, then we assume that $G$ is the
  automorphism group rather than the simple group. Now we can obtain
  generators for the simple group immediately using explicit words in
  the automorphism group standard generators, given in
  \cite{atlas_www}. For a sporadic group, the simple group has index
  at most $2$ in its automorphism group, so in this case the maximum
  number of coset representatives is $2$. On the other hand, if we
  fail to find standard generators of the automorphim group, we assume
  that $G$ is in fact already a central extension of the sporadic
  group, and hence the node is a leaf.
\item If $D$ was named as an alternating group, there are again at
  most $2$ coset representatives.
\item Since the naming algorithm provides the defining field of $G$,
  we know if $G$ is a representation in defining characteristic. In
  this case a conjecture\footnote{Scott Murray, personal
    communication} states that $\abs{G : D} \leqslant \gcd(q - 1, m)
  e$ where $m$ is the maximum sum of the highest weights of $G$, and
  $e = 2$ if $G$ is an orthogonal group and $1$ otherwise. Since
  $\abs{G : D} \geqslant \abs{E}$ this provides an upper bound on the
  number of coset representatives. Moreover, we can calculate the
  maximum highest weight sum for representations of degree up to
  $250$ using \cite{MR1901354}.
\end{itemize}
If we learn that the maximum number of cosets is $1$, then we know
immediately that the node is a leaf, and when enumerating coset
representatives, we can stop when we have reached the maximum.

The kernel of the map onto the coset representatives is found using
the general methods.
\end{enumerate}

\section{Centre calculations}
\label{section:c9_centre}
In the C9 context, and in leaves, we need to obtain generators for
$\Zent(G)$ as $\SLP$s in the generators of $G$. In fact, since $G$ is
absolutely irreducible and contained in $\SL(d, q)$, the centre
consists of scalar matrices and is therefore cyclic. We require a
single centre generator. 

We calculate the centre using the Monte Carlo algorithm \cite[Theorem
4.15]{MR1829470}. In this context we know that an upper bound on the
order is $\gcd(d, q - 1)$. Having found a generating set $X$ for the
centre, we can calculate orders and take powers of the elements of $X$
to obtain elements $c_i$ of order $p_i^{e_i}$, where $\abs{\gen{X}} =
\prod_{i = 1}^n p_i^{e_i}$. Our single generator is then $c = \prod_i c_i$.

\section{Scalar flags}
\label{section:scalars}

As has been indicated above, in some of the Aschbacher reductions we
only obtain homomorphisms modulo scalars. Superficially this is only a
detail, but it has implications for the whole composition tree design, 
and it complicates matters.

In the tensor product case, we have a group $G \leqslant \GL(d, q)$
which preserves a tensor decomposition $V \cong U \otimes W$ of its
natural module. This means that $G \cong H_1 \circ H_2$. The point is
that $G$ is a central rather than a direct product, and the
homomorphism is not $G \to H_1$ but $G \to H_1 / \Zent(H_1)$, so the
image lives in $\PGL(U)$ not $\GL(U)$. However, we have no means of
writing down a quotient of matrix groups, so we need to stay with
$H_1$ as the image and somehow perform the quotient indirectly.

As a consequence, the tensor decomposition algorithm
\cite{tensorprodalg} which constructs the image of $G = \gen{X}$ under
the tensor decomposition homomorphism $\phi$, only guarantee that the
image $H = \gen{\phi(X)} \leqslant H_1 Z$ where $Z = \Zent(\GL(U))$.
Moreover, this is true for every element $g$ that we map under $\phi$.
It is only guaranteed that $\phi(g) \in H_1 Z$, not necessarily that
$\phi(g) \in H_1$ and not even $\phi(g) \in H$. The actual image group
of $G$ is not $H$ but $H / Z = H_1 Z / Z \cong H_1 / (H_1 \cap Z) =
H_1 / \Zent(H_1)$.

There is a similar behaviour in the smaller field, when the group $G
\leqslant \GL(d, \F_q)$ can only be written over a smaller field
$\F_s$ modulo scalars. The algorithm of \cite{smallerfield} constructs
a homomorphism $G \to \F_q^{\times} / \F_s^{\times}$, but each computed image
is only guaranteed to lie in $\F_q^{\times}$

The method we have used to work around these obstacles is that each
node in the composition tree with group $G$ has an associated
\emph{scalar flag} $\lambda$. The whole issue only arises for matrix
groups, so if $G$ is a permutation group then $\lambda = 1$, but if $G
\leqslant \GL(d, q)$ then $\lambda \in \Zent(\GL(d, q))$, a scalar
matrix. Now the matrix group we can work with in the node is $G$, but
the group that the node really represents is $G \gen{\lambda}
/\gen{\lambda}$. Hence the composition factors corresponding to the
node are those of $G \gen{\lambda} / \gen{\lambda}$. The rewriting
algorithm in the node must be able to take as input elements of $G
\gen{\lambda}$, and the rewriting of $g \in G \gen{\lambda}$ is
allowed to return an $\SLP$ that in $G$ evaluates to $g z$ for some $z
\in \gen{\lambda}$.

Every algorithm for processing a leaf must now be able to behave like
this. For every reduction, we also have to determine what the scalar
flags for the image and kernel nodes are, given the scalar flag in the
node.

All Aschbacher reduction algorithms are applied to $G = \gen{X}$, but
the C8 reductions are applied to $G \gen{\lambda}$. This is because
the C8 reductions will give different answers when applied to $G$ and
$G \gen{\lambda}$. The input generators of the image are still the
images of $X$, because we must maintain the bijection between these.

\subsection{Scalar flag propagation}
Let $\omega_q$ be a primitive element of $\F_q$. Assume the
node group is $G$ with scalar flag $\lambda$, $\phi$ is the reduction
homomorphism, $I$ is the group of the image and $K$ is the group of
the kernel. Let $\lambda_I$ and $\lambda_K$ be the scalar flags of the
image and kernel. These are defined as in Table
\ref{tbl:scalar_propagation}.

\begin{table}[ht!]
\begin{tabular}{c|c|c}
Reduction & $\lambda_I$ & $\lambda_K$ \\
\hline
Unipotent & $\phi(\lambda)$ & $1$ \\
MeatAxe & $\phi(\lambda)$ & $1$ \\
Absolute reducibility & $\phi(\lambda)$ & $\lambda$ \\
Imprimitivity & $\phi(\lambda)$ $(=1)$  & $\lambda$ \\
Semilinearity & $\phi(\lambda)$ $(=1)$  & $\lambda$ \\
Tensor product & $\omega_q$  & $\lambda$ \\
Tensor induction & $\phi(\lambda)$ $(=1)$  & $\lambda$ \\
Extraspecial & $\phi(\lambda)$ & $\lambda$ \\
Smaller field & $\phi(\lambda)$ & $\lambda$ \\
Smaller field modulo scalars & $\omega_q^{\abs{\omega_q} / \lcm(\abs{\lambda}, \abs{\omega_s})}$ & $\lambda^{\abs{\lambda} / \gcd(\abs{\lambda}, n)}$ \\
Determinant & $\phi(\lambda)$ $(=\lambda^n)$ & $\lambda^{\abs{\lambda} / \gcd(\abs{\lambda}, n)}$ \\
Form action & $\phi(\lambda)$ $(=\lambda^n)$ & $\lambda^{\abs{\lambda} / \gcd(\abs{\lambda}, n)}$ \\
Spinor norm & $\phi(\lambda)$ $(\in \set{0, 1})$ & $\lambda^{2 \lambda_I}$ \\
C9 (coset action) & $\phi(\lambda)$ $(=1)$ & $\lambda$
\end{tabular}
\label{tbl:scalar_propagation}
\caption{Scalar flag propagation}
\end{table}

In the smaller field case, $\F_s < \F_q$ is the smaller field. We define $n$ as follows.
\begin{itemize}
\item In the smaller field case, $n = \abs{\omega_s} = s - 1$.
\item In the determinant case, $n = d$.
\item In the form action case, non-unitary case, $n = 2$. In the unitary case, $n = \sqrt{q} + 1$ ($q$ is a square in this case). 
\end{itemize}

Most of the entries in the table comes from the observation that we
have a homomorphism $G \to I$, and we define the induced homomorphism
from $G / (G \cap \gen{\lambda})$. The image of this is $I / (I \cap
\gen{\phi(\lambda)})$ so $\lambda_I = \phi(\lambda)$. In the tensor
product case, as described above, the image is a projective matrix
group, so we must divide out by the full scalar subgroup. In the
smaller field case, the image are the scalars in the larger field
modulo the scalars in the smaller field. Hence we divide out by the
smaller scalars, except that there already be a scalar flag which we
need to take into account. The kernel is a projective matrix group
over the smaller field, so we divide out by the part of the scalar
flag which lies in the smaller field.

In many cases we know explicitly what $\phi(\lambda)$ is. If the image
is a permutation group, the scalar flag will be the identity, because
in each of these reductions a scalar will lie in the kernel of the
homomorphism. It is easy to see that in the C8 cases, $\phi(\lambda) =
\lambda^n$ where $n$ is as above.

One can note that $\lambda_K$ is not always guaranteed to lie in $K$.
This is ok within our design, since the actual kernel group is $K
\gen{\lambda_K} / \gen{\lambda_K}$. However, in the C8 cases we must
make sure that $\lambda_K \in K$, because we will apply the C8 tests in
the kernel to $K \gen{\lambda_K}$. Hence the same C8 test may be
succesful in the kernel if $\lambda_K \notin K$, which will lead to an
infinite loop. 

In the MeatAxe cases, it is easy to see that the kernel cannot contain any scalars, so the scalar flag is the identity.

\subsection{Cyclic group element patches}
As noted above, a leaf must be able to handle the case when there is a
non-trivial scalar flag. From some of the reductions we obtain a
cyclic image, which will be a leaf. The following situation arise in all these.

Let $G = \gen{X}$ be the node group, $\phi : G \to I = \gen{Y_I}$ the
reduction and $K = \Ker(\phi) = \gen{Y_K}$. Assume that $I$ is cyclic
and let $\lambda$, $\lambda_I$ and $\lambda_K$ be the scalar flags for
$G$, $I$ and $K$.

Consider $g \in G \gen{\lambda}$ which we want to express as an $\SLP$
in $X$. The first step is to express $\phi(g)$ as an $\SLP$ in $Y_I$.
However, it is not guaranteed that $\phi(g) \in \gen{Y_I}$, only that
$\phi(g)z_I \in I$ for some $z_I \in \gen{\lambda_I}$. Moreover, if we
obtain an $\SLP$ $w$ of $\phi(g) z_I$, the next step is to evaluate
$w$ on $X$ to obtain an element $h$ and then obtain an $\SLP$ of $g/h$
in $Y_K$. But again, it is not guaranteed that $g/h \in \gen{Y_K}$.
This is ok, we only require that $g / h \in K\gen{\lambda_K}$, but
unfortunately this may not be true either.

To be more precise, if $g = \bar{g} \lambda^i$, where $\bar{g} \in G$,
then $\phi(g) = \phi(\bar{g}) \lambda_I^j$. This follows from the
definition of $\lambda_I$ in Table \ref{tbl:scalar_propagation}. Note
that we know neither $i$ nor $j$, and we want to avoid computing them,
as this is a discrete logarithm computation. We obtain an $\SLP$ for
$\phi(\bar{g})$, so $h = k \bar{g}$ for some $k \in K$. Hence $g / h =
k^{-1} \lambda^i$, but we see from Table \ref{tbl:scalar_propagation}
that it is possible that $\lambda^i \notin \gen{\lambda_K}$ and hence
$g / h \notin K \gen{\lambda_K}$.

What is needed is therefore an algorithm that, given $I$ and
$\phi(g)$, finds $z_I \in \gen{\lambda_I}$ such that $\phi(g) z_I \in
I$. Moreover, given $\lambda$ it should also find $z \in
\gen{\lambda}$ such that $g z / h \in K\gen{\lambda_K}$.

\begin{pr}
\label{pr:scalar_patch}
Let $C, \gen{\lambda} \leqslant \Cent_m$ and consider $x \in
C\gen{\lambda}$. Given the list of primes dividing $m$, in polynomial
time we can find $z \in \gen{\lambda}$ such that $xz \in C$.
\end{pr}
\begin{proof}
Let
\begin{align*}
\abs{C} &= \prod_{i = 1}^n p_i^{a_i}, \\
\abs{c} &= \prod_{i = 1}^n p_i^{b_i}, \\
\abs{\lambda} &= \prod_{i = 1}^n p_i^{c_i}
\end{align*}
where each $p_i$ is prime, $a_i, b_i, c_i \geqslant 0$, $a_i + b_i +
c_i > 0$ and $b_i \leqslant \max(a_i, c_i)$. Let $q_i = \abs{x} /
p_i^{b_i}$ for $i = 1, \dotsc, n$, and observe that $\gcd(q_1, \dotsc,
q_n) = 1$. Using the extended Euclidean algorithm, we can find integers $m_i$ such that $1 = \sum_{i = 1}^n q_i m_i$.

Let $z_i = x^{m_i q_i}$ for $i = 1, \dotsc, n$. Then
  \[ \prod_{i = 1}^n z_i = \prod_{i = 1}^n x^{m_i q_i} =
  x^{\sum_{i = 1}^n m_i q_i} = x, \]
and also $\abs{x^{q_i}} = p_i^{b_i}$. Since $\abs{z_i} =
  (x^{q_i})^{m_i}$ it follows that $\abs{z_i} = p_i^{b_i}$
  unless $\gcd(\abs{x^{q_i}}, m_i) > 1$. But the latter implies
  that $p_i \mid m_i$ and since $p_i \mid q_j$ for every $j \neq i$ it
  follows that $p_i \mid 1$, a contradiction. Therefore $\abs{z_i} =
  p_i^{b_i}$.

Now let
\[ z^{-1} = \prod_{b_i > a_i} z_i \] and observe that if $b_i > a_i$
then $b_i \leqslant c_i$. This implies that $\abs{z} \mid
\abs{\lambda}$ and hence $z \in \gen{\lambda}$.

Finally, $\abs{x z} = \prod_{b_i \leqslant a_i} z_i$, and since
$\abs{z_i} = p_i^{b_i}$ it follows that $x z \in C$. Note that since
we know a superset of the primes $p_i$, there is no need to factorise
$\abs{C}$ and $\abs{c}$.
\end{proof}

Note that the result provides an algorithm for obtaining the patch
element $z_I$. In this case, the cyclic groups
under consideration all arise from some $\F_q^{\times}$, so we can
factorise $q - 1$ once as preprocessing to obtain the list of possible
primes.

The element $z_I$ is unique up to multiplication by an element of $I
\cap \gen{\lambda_I}$. It remains to find the other patch element $z
\in \gen{\lambda}$. This is done differently depending on how
$\lambda_I$ arose from $\lambda$. 

\subsubsection{Determinant case}
In this case, $\lambda_I = \lambda^d$, so $\lambda^i$ is a $d$-th root
of $z_I^{-1}$. However, not all roots suffice. Clearly, all $d$th roots
of $z_I$ are $\lambda^{-i} \sigma$ where $\sigma^d = 1$. Now $g / h
\lambda^{-i} \sigma = k^{-1} \sigma$, so we must choose a root of
$z_I$ such that $\sigma \in \gen{\lambda_K}$. 

We have no knowledge of $\lambda^i$ or $\sigma$, only their product,
which is the root of $z_I$. From Table \ref{tbl:scalar_propagation} we
see that $\abs{\lambda_K} = \gcd(d, \abs{\lambda})$. Also,
$\abs{\lambda^i \sigma} \geqslant \abs{\sigma} / \gcd(\abs{\sigma}, \abs{\lambda})$ so if $\abs{\lambda^i \sigma} \mid \abs{\lambda}$, then $\abs{\sigma}
\mid \abs{\lambda}$. But $\abs{\sigma} \mid d$, so this implies that
$\abs{\sigma} \mid \abs{\lambda_K}$, thus $\sigma \in \gen{\lambda_K}$.

The crucial condition on the $d$th root $z$ of $z_I$ that we choose is
therefore that $z \in \gen{\lambda}$.

\subsubsection{Form action case}
In this case it depends on the form type. In the non-unitary case,
$\lambda_I = \lambda^2$, so we choose $z$ to be a square root of
$z_I$. The argument why this works is the same as in the determinant case.

The unitary case is more difficult. Recall that the \emph{norm} of
$\F_q$ over $\F_{\sqrt{q}}$ is a homomorphism $\Norm : \F_q^{\times} \to \F_{\sqrt{q}}^{\times}$ defined by $\omega_q \mapsto \omega_q^{\sqrt{q} + 1}$.

In this case $\lambda_I = \Norm(\lambda)$, so we want to choose $z \in
\F_{\sqrt{q}}^{\times}$ such that $\Norm(z) = z_I$. This is a norm
equation, which we can solve \cite[Proposition 2.2]{spinornorms08} and
obtain a solution $z_0$. Again, the crucial condition is $z \in
\gen{\lambda}$, but we may have $z_0 \notin \gen{\lambda}$. Each
solution to the norm equation can be written in the form $z_0 z_1$
where $\Norm(z_1) = 1$. Clearly, $\abs{z_0 z_1} \geqslant \abs{z_0} /
\gcd(\abs{z_0}, \abs{z_1})$, and since $\Norm(z_1) = 1$ it follows
that $\abs{z_1} \mid (\sqrt{q} + 1)$. Hence $\abs{z_0 z_1} \geqslant
\abs{z_0} / \gcd(\abs{z_0}, \sqrt{q} + 1)$.

There exists at least one solution $z \in \gen{\lambda}$ to the norm
equation, so the above implies that $\abs{z_0} / \gcd(\abs{z_0},
\sqrt{q} + 1)$ divides $\abs{\lambda}$. Therefore $\abs{z_0} \mid \lcm(\abs{\lambda}, \sqrt{q} + 1)$.

Thus, $z_0 \in \gen{\lambda, \omega_q^{\sqrt{q} - 1}}$, since
$\abs{\gen{\omega_q^{\sqrt{q} - 1}}} = \sqrt{q} + 1$, and we want to find
$z_1 \in \gen{\omega_q^{\sqrt{q} - 1}}$ such that $z_0 z_1 \in
\gen{\lambda}$. This is precisely the same situation we had above,
when we were finding $z_I$. Therefore we can use the patching method
of Proposition \ref{pr:scalar_patch} to obtain $z_1$.

\subsubsection{Spinor norm case}
In this case, $\lambda_I \in \set{0, 1}$. Accordingly, $z = \lambda^{z_I}$.

\subsubsection{Smaller field case}
In this case there is no need for a second patch element, so $z = 1$.

\section{Reduction database for permutation groups}
\label{section:perm_reductions}

Our goal here is to provide reductions from general permutation groups
to small base groups, which can be treated as leaves using BSGS
methods. We do not quite do this, since we are missing the \lq\lq
jellyfish'' algorithm.

\subsection{Intransitivity}
If $G \leqslant \Sym{n}$ is intransitive, then $\mathcal{X} = \set{1,
  \dotsc, n}$ is split up into at least two orbits under the action of
$G$. If $\mathcal{O} \subset \mathcal{X}$ is such an orbit, there is
an induced action of $G$ on $\mathcal{O}$. It is straightforward to
construct $\phi : G \to \Sym{\mathcal{O}}$. We have no special method
of obtaining the kernel.

\subsection{Imprimitivity}
Similarly as with matrix groups, a transitive $G \leqslant \Sym{n}$ is
imprimitive if it permutes the blocks in a non-trivial partition of
$\mathcal{X} = \set{1, \dotsc, n}$. Here $\MAGMA$ provides all
machinery for setting up the reduction homomorphism. We have no
special method of obtaining the kernel.

\subsection{Alternating groups natural representation}
Another large base case are the alternating group in natural
representation. We use the existing algorithm in $\MAGMA$ to see if $G
= \Alt(n)$, and if so it is marked as a leaf, and handled using our
special algorithm.

\section{Leaf algorithm database}
\label{section:leaf_algorithm}

When we have a leaf, we first look at the type of the group. If it is
not a matrix or a permutation group, we handle it accordingly.
Otherwise, the group has either been named during the C9 reduction, or
it is an unknown group.

If it has been named, we consider the list of possible leaf algorithms
for the family, and try each one in turn until one succeeds.

The library of algorithms for handling the various types of leaves is
in a sense the most critical component of the package and the one
where most improvements are needed. 

\subsection{Cyclic groups}
The cyclic leaves have been discussed above. In $\MAGMA$ they arise as
\texttt{GrpAb}, a finitely presented group, and are stored as such. It
would have been slightly easier if they could be converted into
PC-groups, but this is not possible if the order is larger than
$2^{30}$.

The gold copy of a cyclic group is the group itself. $\MAGMA$ always
introduces a minimal generating set for a \texttt{GrpAb}, in this case
a single generator, and we take the nice generator to be this
generator.

A presentation is straightforward to obtain: a single relation is
required. If $G$ is the group with scalar flag $\lambda$, we can
easily form $H = G\gen{\lambda} / \gen{\lambda}$ since the groups are
finitely presented. The presentation is then $\gen{x | x^{\abs{H}}}$.

The composition factors are also straightforward to obtain using
existing $\MAGMA$ functionality.

\subsection{Elementary abelian groups}
These arise from the unipotent groups, and we store them in $\MAGMA$
as \texttt{GrpPC}, a finitely presented group where the presentation
is a power-commutator (PC) presentation. For \texttt{GrpPC} one can
define homomorphisms slightly more freely. This is useful here, since
$\MAGMA$ introduces a new generating set, which
we take as nice generators, and we need to express these as words in
our input generators. The gold copy is the group itself, on the nice
generators.

A presentation for a \texttt{GrpPC} is easy to obtain, since the group
is already stored as a presentation. Since we have chosen the nice
generators to be the $\MAGMA$ generators, the stored presentation is
the one we need. 

The composition factors are straightforward to obtain using
existing $\MAGMA$ functionality.

\subsection{Classical groups natural representation}
These leaves are quasi-simple matrix groups, the families $\SL(d, q)$,
$\Sp(d, q)$, $\SU(d, q)$ and $\Omega^{\epsilon}(d, q)$. They form the leaves in the C8 Aschbacher class. If $G =
\gen{X} \leqslant \SL(d, q)$ is a leaf group of this type, we use the
algorithms in \cite{classical_recognise}, with implementations by
Eamonn O'Brien, to find standard generators $Y$ for $G$. This also
provides a change of basis matrix $c \in \GL(d, q)$ such that
$\gen{Y}^c = G$. We define the gold copy of $G$ to be $\gen{Y}$, so
that $c$ also gives us isomorphisms to and from the gold copy. The
nice generators are $Y^c$.

The algorithms in \cite{costi_phd}, with implementation by Elliot
Costi, provides constructive membership testing in the nice
generators.

We have presentations for the classical groups on the standard
generators, hence on our nice generators.

The composition factors consist of the simple group and the factors
corresponding to $\Zent(G)$. In this case we do not write down the
simple group explicitly, since it is a permutation group of large
degree. The generators of the simple group factor is instead just the
nice generators, so the $\SLP$s are immediate. Since $G$ is a
classical group in natural representation, we can write down an
explicit scalar matrix that generates $\Zent(G)$. Then we can obtain
this as an $\SLP$ in the standard generators using the rewriting
algorithm. The centre can easily be converted to a PC-group, we can
take the quotient with the scalar flag, and then use existing
$\MAGMA$ functionality to obtain the composition factors.

Let $\omega$ be a primitive element of $\F_q$. Then $\Zent(G) = \gen{\omega^{(q - 1) / n}}$ where 
\[n = \begin{cases}
\gcd(d, q - 1) & G = \SL(d, q) \\
\gcd(2, q - 1) & G = \Sp(d, q) \\
3 & G = \SU(3, 4) \\
\gcd(d, \sqrt{q} + 1) & G =�\SU(d, q), (d, q) \neq (3, 4) \\
1 & G = \Omega^{\epsilon}(d, q), 2 \mid q \\
1 & G = \Omega^0(d, q) \\
2 & G = \Omega^{+}(d, q), 2 \mid (d(q - 1) / 4) \\
1 & G = \Omega^{+}(d, q), 2 \nmid (d(q - 1) / 4) \\
2 & G = \Omega^{-}(d, q), 2 \nmid (d(q - 1) / 4) \\
1 & G = \Omega^{-}(d, q), 2 \mid (d(q - 1) / 4) 
\end{cases} \]

\subsection{Quasi-simple groups, C9}
The C9 leaves are the most difficult ones, and those where we have
fewest special leaf algorithms. We do have access to a number of
algorithms, but not all work in the context we need.

In the C9 leaf case we have $G \leqslant \GL(d, q)$, with scalar flag
$\lambda$, and $G \cong Z.S$ where $S$ is a finite simple group and
$Z$ is cyclic. Since $G$ is absolutely irreducible, $Z \leqslant
\Zent(\GL(d, q))$ and hence consists of scalar matrices. Since we have
applied the determinant map, $G \leqslant \SL(d, q)$, so $\abs{Z} \mid
\gcd(d, q - 1)$. This is the only thing we can say about $Z$ in the
general case. The extension $Z.S$ may split, it can even be a direct
product, so $Z$ is not guaranteed to lie in the Schur multiplier of
$S$. Hence $G$ is not necessarily perfect.

We now want the leaf algorithms to handle such a $G$. Hence a leaf
algorithm should provide maps $\alpha : G\gen{\lambda} \to H$, where
$H \cong S$ or $H \cong Z_S . S$ is the gold copy. Here $Z_S$ are the
scalars in the Schur multiplier of $S$, and we need to allow this
situation, to correctly handle the case when $S \cong \PSL(d, q)$.
Then we want $H = \SL(d, q)$ because we want to work with matrix
groups.

The leaf algorithm should also provide $\beta : H \to G$, which should
be the inverse modulo scalars to $\alpha$. In other words,
$\beta(\alpha(g)) \in Z\gen{\lambda}$ for every $g \in G\gen{\lambda}$.

We need a notion of standard generators for $G$. This is given
implicitly by the images under $\beta$ of the generators of $H$. Then
the leaf algorithm should provide a rewriting facility that expresses
any $g \in G\gen{\lambda}$ as an $\SLP$ in the standard
generators, which evaluates to $h \in G\gen{\lambda}$ with the
property that $g / h \in Z\gen{\lambda}$.

Our nice generators for $G$ are the standard generators together with a
generator for $Z$. We can find the centre of $G$ as described in
Section \ref{section:c9_centre}. This allows us to obtain a rewriting
algorithm for $G$ which is precise. We use the above leaf algorithm
rewriting, so we obtain an element $x \in Z \gen{\lambda}$. If $Z =
\gen{c}$ then $\abs{x} \mid \lcm(\abs{c}, \abs{\lambda})$. Observe
that $y = x^{\abs{x} / \gcd(\abs{c}, \abs{x})} \in Z$, so we can
express $y$ as an $\SLP$ in $c$. This is an instance of the discrete
logarithm problem.

Finally, we want the leaf algorithm to provide a presentation of $S$
on the standard generators. This allows us to obtain a presentation
for $G$. The presentation $\gen{x_1, \dotsc, x_n \mid r_1, \dotsc,
  r_m}$ for $S$ is a presentation for $G/Z$. We want to obtain a
presentation of $G$ on $\set{x_1, \dotsc, x_n, c}$. Each $r_i$
evalutes to $c_i \in Z$, and we can express each $c_i$ as an $\SLP$
$w_i$ in $c$, as above. The relators for $G$ are then $\set{r_i / w_i
  : 1 \leqslant i \leqslant m} \cup \set{[x_i, c] : 1 \leqslant i \leqslant n} \cup \set{c^{\abs{c} / \gcd(\abs{c}, \abs{\lambda})}}$.

To obtain composition factors in the C9 case is easy, since we can
find the centre. We then proceed as in the C8 case.

\subsubsection{Alternating groups}
In this case we use \cite{MR1743388} with implementation by Derek
Holt, extended by Eamonn O'Brien. This works exactly as we require.

\subsubsection{Sporadic groups}
For the sporadic groups we have to proceed in a slightly different
way. For each sporadic group we designate a small-dimensional
permutation or matrix representation as the gold copy $S$. Then we use
the black box algorithms from \cite{atlas_www} to obtain standard
generators in $G$ and in $S$. These algorithms work projectively and
hence can be applied to $G$ without knowledge of $Z$. 

To obtain a rewriting algorithm, we either use \cite{ryba_trick} or
BSGS methods. This also allows us to define the maps $\alpha$ and
$\beta$: express an element as an $\SLP$ in the standard generators in
one group and evalute the $\SLP$ on the standard generators in the
other group.

For most sporadic groups we also have presentations on the standard
generators, but in some cases we must use brute force methods.

\subsubsection{Classical groups non-natural representation}
This is the most difficult class of leaves. In the case where $S \cong
\PSL(2, q)$ we use \cite{psl_recognition}, with implementation by
Eamonn O'Brien, which works exactly as we require.

In the case where $S \cong \PSL(3, q)$, and $G$ is a representation in
defining characteristic, we use \cite{MR2356848}, with implementation
by Henrik B��rnhielm and Eamonn O'Brien, to obtain the map $\beta$.
This algorithm cannot handle the case when $Z$ is not contained in the Schur multiplier of $G$. We work around
this by computing the derived group $D \cong S$ of $G \gen{\lambda}$
and executing the algorithm on $D$. The map $\beta$ allows us to
obtain standard generators for $G$, and we define the nice generators
as above. From \cite{MR2356848} we obtain rewriting ability in $D$,
and all the standard generators lie in $D$, so can obtain them as
$\SLP$s in the input generators of $D$ and hence in the input
generators of $G$. For rewriting in $G\gen{\lambda}$ we use
\cite{costi_phd}, with implementation by Elliot Costi, which works as
we require. This also provides the map $\alpha$, in the same way as
with the sporadic groups.

When $S \cong \PSL(d, q)$, $d > 3$ and $G$ is a representation in
defining characteristic, we use either \cite{small_deg_reps}, with
implementation by Eamonn O'Brien or \cite{MR1804385} with
implementation by Peter Brooksbank, Bruce Cox and Derek Holt, to
obtain $\beta$. Similarly as above we first have to compute the
derived group $D$ of $G\gen{\lambda}$. In this case we have no
rewriting ability in $D$. However, we obtain $\alpha_0 : D \to \SL(d,
q)$, so we can obtain the images of our input generators in the
natural representation. Then we can use \cite{classical_recognise} to
obtain the standard generators as $\SLP$s in the input generators. For
rewriting in $G\gen{\lambda}$ we can again use \cite{costi_phd}, which
also provides $\alpha$.

When $G$ is not in defining characteristic, or if $G$ is a permutation
group, then we use \cite{MR1804385}. We can obtain the nice generators
as $\SLP$s in the input generators as above. To obtain rewriting
ability in $G\gen{\lambda}$, we use BSGS methods or \cite{ryba_trick}.
Since the map to $\SL(d, q)$ is from $D$, not $G\gen{\lambda}$, we
cannot use it to map elements to the natural representation and use
our natural representation rewriting ability.

When $S \cong \PSp(d, q)$ and $q$ is odd, we use \cite{MR1958954,
  MR2422320} with implementation by Peter Brooksbank and Derek Holt.
This works in the same way as the black box $\PSL(d, q)$ case.

In all these cases we have presentations on the standard generators.

\subsubsection{Exceptional groups}
The algorithms of \cite{baarnhielm05, baarnhielm_phd,
  small_ree_recog}, with implementation by Henrik B��rnhielm, handle
the cases of Suzuki and Ree groups in most ways we require. The
representations of the Suzuki groups in defining characteristic
(characteristic $2$) have dimension $4^n$ for some $n \geqslant 1$, so
$\abs{Z} \mid \gcd(4^n, 2^k - 1)$ and hence $Z = 1$. In black box
context the Suzuki algorithms cannot handle the case when $Z \neq 1$.

The algorithms for the Big Ree groups only handle the natural representation,
dimension $26$, with defining field of size $q = 2^{2m + 1}$ for some $m > 0$.

The algorithms for the small Ree groups only handle representations in
defining characteristic (characteristic $3$). Such representations
have dimension $7^n 3^{3k}$ where $n \geqslant 0$, $k \geqslant 0$ and
$n + k > 0$.

It follows from the following result that $Z = 1$ in the Ree group cases.

\begin{lem}
\begin{enumerate}
\item For every $m > 0$, $13 \nmid (2^{2m + 1} - 1)$.
\item For every $k > 0$, $7 \nmid (3^k - 1)$.
\end{enumerate}
\end{lem}
\begin{proof}

\end{proof}

In the Suzuki case we have a presentation on the standard generators,
but we have no presentations for the Ree groups, so in these cases we
are forced to use existing $\MAGMA$ machinery.

\subsection{Unknown groups}
Because we do not have special algorithms for all classes of finite
simple groups, it is inevitable that we need a backup strategy for
groups we cannot handle. It is also possible that some Aschbacher
algorithms fail to deduce that a give group lies in its category,
which can result in that it falls into the C9 case. But then the
naming algorithms may fail, since they assume that the group is almost
simple modulo scalars. Hence we end up with a group which really is
unknown. 

In the first of these cases we either use BSGS techniques or
\cite{ryba_trick} together with recursive applications of composition
tree. In the second case we always use BSGS methods.

More specifically, if $G = \gen{X}$ is the input group with scalar
flag $\lambda$, and we have successfully named $G$, then we use
\cite{ryba_trick} if the field or degree is larger than some specified
bound. In this case the gold copy is $G$ and $X$ are the nice
generators. For membership testing we apply \cite{ryba_trick} to $G
\gen{\lambda}$, and membership testing in the centralisers are done
using recursive applications of composition tree. Note that we then
obtain $\SLP$s in $X \cup \set{\lambda}$, so we have to rewrite this
to an $\SLP$ on $X$ (we set $\lambda = 1$). Presentations and
composition factors are found using existing $\MAGMA$ machinery, which
may be very expensive.

If we cannot name $G$, or the the field and degree are small, we
compute a BSGS of $G \gen{\lambda}$ using the random Schreier-Sims
algorithm and possibly verify it using the Todd-Coxeter-Schreier-Sims
algorithm. Then $\MAGMA$ provides constructive membership testing
facilities in $G \gen{\lambda}$. As above, we need to rewrite the
$\SLP$s to obtain them on $X$, without $\lambda$. The gold copy is
$G$, and the nice generators are the strong generators. We can find a
presentation on the strong generators, which is easier than in
general, but can still be expensive. Composition factors are found
using existing $\MAGMA$ machinery, which can be very expensive.

\bibliographystyle{amsalpha}
\bibliography{mgrp}

\end{document}
