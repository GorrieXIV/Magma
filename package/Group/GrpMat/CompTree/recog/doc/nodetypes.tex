\magnification 1200
\def\q{\quad}
\def\GL{{\rm GL}}
\centerline{\bf Data types for composition tree 1.}
\medskip
\centerline{Types of nodes}
Matrix; Cyclic; Permutation.

Cyclic subtypes: 

\q Abstract

\q Field
\medskip
Matrix subtypes:

\q General

\q SemiSimple

\q Decomposable

\q Unipotent, with subtypes

\q \q Compound

\q \q Single

\q Irreducible, with subtypes

\q\q Classical

\q\q Larger field

\q\q Smaller field

\q\q Imprimitive

\q\q Tensor product

\q\q Tensor Induced

\q\q Symplectic

\q\q C9

\q Block

\q Subgroup
\medskip
\centerline{Comments}
\medskip
Some types of node I regard as leaves from the point of view of this essay.  Constructing
composition factors for these nodes is another matter.  Call such nodes `pre-leaves'.
These are the types Cyclic, Permutation, Matrix--Unipotent--Single, Matrix--Irreducible--Classical,
and Matrix--Irreducible--C9.

Each type of node that is not a pre-leaf has a right and a left child of prescribed type (with
the exception of subgroup, whose right child can be of two types).  But any node can
turn into the type of either of its children.  Once a node has been completed its type cannot change.

Cyclic groups.  Distinguish between cyclic groups given as abstract groups, and hence 
determined by a generator and its order, and groups given as sections of the multiplicative
group of a field, where elements are defined by field elements.

Matrix groups.

General.  For example, the root.  Universal group $\GL (d,q)$.

SemiSimple.  Universal group, $\prod_i\GL(d_i,q)$; the groups $\GL(d_i,q)$ are partitioned, to
represent a decomposition of the original module $V$ into direct summands.

Decomposable.  Universal group $\GL(d,q)$.  The group is known to act completely reducibly
(or semi-simply), but no decomposition is imposed.

Unipotent Compound.  Any unipotent group; it will come in blocks.

Unipotent Single.  An elementary abelian section of a unipotent group appearing as a single
rectangular block.

Irreducible.  Subtypes depend on Aschbacher classification.

Block.  A block arises first from imprimitivity, or from tensor induced, as the left child.  
When the image of the group on the first factor has been determined, the action of the kernel 
on the second block is
(up to conjugation by a fixed element) isomorphic to a subgroup of the action on the first block,
and so forth.  So there are eighteen types of node.
\bye


