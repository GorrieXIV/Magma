\magnification 1200
\def\q{\quad}
\def\GL{{\rm GL}}
\centerline{\bf Data types for composition tree 3.}
\medskip
\centerline{Subgroups}
\medskip
Given a matrix group $G$ for which a (geometric) composition tree has been constructed,
and a subset $Y$ of $G$: to construct a composition tree for $H=\langle Y\rangle$, that
relates to the composition tree for $G$.  Allow $G$ to be any node of a larger composition tree.
This node will be the location for the root of $H$.

The final output will be a composition tree that can be described as follows.  Take the composition
tree for $G$ and replace the leaves by composition trees for subgroups of these leaves; and
finally delete the nodes that correspond to trivial groups.  In fact nodes corresponding to
trivial groups will be retained (as place holders) provided that they have a nontrivial parent.

Set up $H$-mandarins at the $H$-root for $G$.  Now proceed recursively.

If the node is a leaf, declare the node to be of type General (for $H$) and construct a composition
tree.

If the node has the property that every generator for $H$ maps to the identity on the right, the
right child is the trivial group (new type of node?), and the generators move down to
the left child.

We now have a node, with a non-trivial map to the right.  Process this node exactly as with the
standard composition tree, and get generators for the kernel.  If these are all trivial the
left node is trivial; if not, proceed as with composition tree. 

The maps between nodes (other than nodes obtained by refining $G$-leaves) are determined by the
$G$-composition tree.  The $H$ composition tree mechanism is not allowed to change the basis
for the underlying vector space.  So composition tree must be able to work with the change of basis mechanism turned off.
\bye