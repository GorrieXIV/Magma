\magnification 1200
\centerline{\bf Data structures for matrix recognition.  General}
\medskip
A node $N$ has associated with it:

A type, consisting of a name and parameters, that define a group $U_N$, the {\it universal}
group for the node:

A group $G_N$  isomorphic to $U_N$, and computable isomorphisms $i_N$ from
$G_N$ to $U_N$ and $i_N^{-1}$.

A function that returns generators for a subgroup $H_N$ of $G_N$.  This will usually be defined
by a list $X_N$ of generators; but in one case $H_N$ is a normal subgroup of a larger group, and the 
generators listed are normal subgroup generators, so that the function in this case is slightly more
complex.

A function that returns mandarins as elements of $H_N$.

The name of the group can be of simple or compound type.

If the name is of compound type a normal subgroup $K_N$ of $U_N$ is defined by the name and
parameters of the type.  In this case the node has a right child $N_1$, with $U_{N_1}=U_N/K_N$,
and a left child $N_0$ with $U_{N_0}=K_N$.  A generator of $N_1$ is defined by taking a
generator of $N_0$, mapping it under $i_N$ to $U_N$, by the natural map to $U_{N_1}$, and
under $i_{N_1}^{-1}$ to $H_{N_1}$.  The mandarins are similarly defined.

When the right child has been completed mandarin and generator functions for the left child are
defined by the familiar process.

If the node is of simple type it will either reduce or form a geometric leaf.

If it reduces, a node $N_A$ ($A$ for `alias') is constructed (with a different name 
and parameters), with the property that $U_{N_A}$ is a subgroup of $U_N$, and an assertion
that $i_N(H_N)$ lies in $U_{N_A}$.  Now $G_{N_A}$ is constructed isomorphic to $U_{N_A}$,
and the various isomorphisms are set in place, and the functions that construct generators and
mandarins are set up.

The node $N$ is supplied with a pointer to $N_A$, and {\it vice versa}.

If a node does not reduce a solution to the explicit membership problem is constructed at the node.

\bye

